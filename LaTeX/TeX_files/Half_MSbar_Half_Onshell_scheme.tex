% mainfile: ../master_thesis_GW.tex
\chapter{Half \MSbar-Half Onshell scheme}
Due to (so far) lack of experimenttal data of coupling $\lambda$ in considered theories, 
the on shell condition for that constatnt renders itself meaningless. \\
Thus, we propose mixed scheme, where we demand that the physical mass remain unchanged due to 
the one-loop corrections, but for the coupling case, we demand only that the $\dl$ 
counterterm is such that the fourth  derivative of the renormalzsed effective potential is finite 
-- in the \MSbar manner.
%\section{Finite momentum}

\section{Zero momentum}
Here we impose following renormalisation conditions:
\begin{align}
\DphiII{}\VRIloop\Big|_{v} &= 0 \\
\DphiIV{}\VRIloop\Big|_{v} &= \frac{9e^4}{8\pi^2}\Big(-\frac{5}{6}+\log\frac{e^2
v^2}{\mu^2}\Big)
+\frac{75e^4}{16\pi^2}
\end{align}
Written in the full form these conditions take form:
\begin{align}
&\frac{9e^4}{16\pi^2}\Big(-\frac{2}{\epsilon}+\gamma_E-\frac{5}{6}+\log\frac{e^2
v^2}{4\pi\mu^2}\Big)v^2+ \notag\\
& 12\cl\dl v^2+\frac{21e^4}{32\pi^2}v^2+2\cm\dm = 0 \\
&\frac{9e^4}{8\pi^2}\Big(-\frac{2}{\epsilon}+\gamma_E-\frac{5}{6}+\log\frac{e^2v^2}{4\pi\mu^2}\Big)+
24\cl\dl+\frac{75e^4}{16\pi^2} = \notag \\
&\frac{9e^4}{8\pi^2}\Big(-\frac{5}{6}+\log\frac{e^2v^2}{\mu^2}\Big)+\frac{75e^4}{16\pi^2}
\end{align}
After solving equations for $\dm$ and $\dl$ we obtain:
\begin{align}
\dm &= -\frac{3e^4v^2}{32\cm\pi^2}\Big(1+3
\log\frac{e^2v^2}{\mu^2}\Big) \\
\dl &= \frac{3}{64\cl\pi^2}\Big(\frac{2}{\epsilon}-\gamma_E+\log(4\pi)\Big)
\end{align}
The renormalized potential is then:
\begin{align}
\VR = &\cl\lambda\varphi_1^4 + \notag \\ 
&\frac{3e^4}{64\pi^2}\Big(-\frac{5}{6}+\log\frac{e^2\varphi_1^2}{\mu^2}\Big)\varphi_1^4+ \notag\\
&\frac{3e^4v^2}{32\pi^2}\Big(-1-3\log\frac{e^2v^2}{\mu^2}\Big)\varphi_1^2
\end{align}
From the tree level potential we have the relation $\lambda = \frac{M_P^2}{12\cl v^2}$. 
Written in these terms we have:
\begin{align}
\VR = &\frac{M_P^2}{12v^2}\varphi_1^4 + \notag \\ 
&\frac{3e^4}{64\pi^2}\Big(-\frac{5}{6}+\log\frac{e^2\varphi_1^2}{\mu^2}\Big)\varphi_1^4+ \notag\\
&\frac{3e^4v^2}{32\pi^2}\Big(-1-3\log\frac{e^2v^2}{\mu^2}\Big)\varphi_1^2
\end{align}
We can bind $e$ to $M_P$ and $v$ at the loop level, from the definition of VEV:
\begin{align}
\DphiI\VR \Big|_v= 0
\end{align}
This gives the condition:
\begin{align}
-\frac{e^4v^3}{4\pi^2}-\frac{3e^4v^3}{8\pi^2}\log\frac{e^2v^2}{\mu^2}+\frac{M_P^2v}{3}=0
\end{align}
Setting scale parameter $\mu$ to the effective mass of the vector, namely $ev$, we have 
simpler form of:
\begin{align}
-\frac{e^4v^3}{4\pi^2}+\frac{M_P^2v}{3}=0
\end{align}
Which gives:
\begin{align}
e^4 = \frac{4M_P^2\pi^2}{3v^2}
\end{align}
Writing potential with this substitutions yields asdad:
\begin{align}
V_R = \frac{M_P^2}{16v^2}\Big(\frac{1}{2}+\log\frac{\varphi_1^2}{v^2}\Big)\varphi_1^4- 
\frac{M_P^2}{8}\varphi_1^2
\end{align}
We can also derive direct relation between $\lambda$ and $e^4$, namely:
\begin{align}
\lambda = \frac{e^4}{16\cl\pi^2}
\end{align}

