% mainfile: ../master_thesis_GW.tex
\chapter{On shell renormalisation of the effective potential}

One of the main topic of this thesis is to show a coherent way to renormalize a conformal theory 
without explicit mass term in the on-shell scheme. \\


To calculate on-shell renormalisation we need to calculate self energy. 
However, it turns out, that simple calculation of self energy fails the test of 
comparison between the zero-momentum limit of the self energy and the second derivative 
of the effective potential. \\
Namely, it schould be safisfied that:
\begin{equation}
\lim\limits_{p^2\to 0}\Sigma(p^2)=\frac{\partial^2V_{eff}}{\partial\varphi_1^2},
\end{equation}
\todo{napisać ile wychodzi}
but it is not the case. \\

However, from the \MSbar considerations, we know that $\Phi$ have non-zero VEV, let us 
call it $v$. 
Let us rotate $\Phi$ in such a way, that $\langle\varphi_1\rangle=v$ and $\langle\varphi_2\rangle 
= 0$, where now $v$ is real. \\
Keeping this in mind, we can rewrite Lagransian in terms of shifted fields $\varphi_1$, 
$\varphi_2$ which have both zero VEV, now VEV is explicitly in the Lagransian:
\begin{align}
\mathcal{L} = -\frac{1}{4}F_{\mu\nu}F^{\mu\nu}+ 
\frac{1}{2}(\partial_\mu(\varphi_1+v) - eA_\mu\varphi_2)^2\\
+\frac{1}{2}(\partial_\mu\varphi_2+eA_\mu(\varphi_1+v))^2
-\cl\lambda((\varphi_1+v)^2+\varphi_2^2)^2. \notag
\end{align}

This breaks the symmetry, but now there are more interaction terms in the Lagransian and this leads 
to different self energy, now consistent with the second derivative of the effective potential, 
as will be shown in \ref{}. \\ 

Following \cite{Coleman1973} we put the mass counterterm even though initialy the mass term 
was not present in the Lagransian. It will turn out to be crutial in \ref{}. \\ 
The Lagransian with $\dZphi$, $\dZl$ and $\dZm$ counterterms looks like this:

\begin{align}
\mathcal{L_R} = &-\frac{1}{4}F_{\mu\nu}F^{\mu\nu} \\
+(1+ \dZ)&(\frac{1}{2}(\partial_\mu(\varphi_1+v) - eA_\mu\varphi_2)^2
+\frac{1}{2}(\partial_\mu\varphi_2+eA_\mu(\varphi_1+v))^2) \notag \\
-(1+\dZ)^2&\cl(\lambda+\dl)((\varphi_1+v)^2+\varphi_2^2)^2  \notag \\
+ \dm &c_m((\varphi_1+v)^2+\varphi_2^2).
\end{align}

%We will write with the convention that:
%\begin{equation}
%\dl=\dZl\lambda
%\end{equation}

Separating the terms with the first power of renormalisation constants and second power of 
$\varphi_1$, we obtain correction to the self energy equal to:
\begin{equation}
\dS=-6\cl v^2(2\lambda\dZphi+\dl)+\cm\dm-p^2\dZphi,
\end{equation}

where $p^2 = \partial_\mu \varphi_1\partial^\mu\varphi_1$.


Separating the terms with the first power of renormalisation constants and first power of 
$\varphi_1$, we obtain correction to the tadpole equal to:
\begin{equation}
\dT=-4\cl v^3(2\lambda\dZphi+\dl)+2v\cm\dm
\end{equation}

First approuch is to impose renormalisation conditions resembling clissical on-shell ($M_P$ 
stands for physical mass):
\begin{align}
T+\dT &= 0 \\
\Rep{\Sigma(p^2=M_P^2)} + \Rep{\dS(p^2=M_P^2)} &= M_P^2 \\
\frac{\textrm{d}\Sigma}{\textrm{d}p^2}(p^2=M_P^2) &= 0
\end{align}

We will denote $\frac{\textrm{d}\Sigma}{\textrm{d}p^2}$ as $\Sigma'$. \\

This gives us:
\begin{align}
\dm &= \frac{1}{4\cm}\left(-\Rep{\Sigma(M_P^2)}+M_P^2+\frac{3}{v}T+M_P^2\Rep{\Sigma'(M_P^2)}
\right)\\
\dl &= \frac{1}{8\cl v^3}\left(v\Rep{\Sigma(M_P^2)}-T-(16\cl\lambda v^3+M_P^2v)\Rep{\Sigma'(M_P^2)}
\right)\\
\dZphi &= \Rep{\Sigma'(M_P^2)}
\end{align}




\section{"Zero momentum" approuch}














