% mainfile: ../master_thesis_GW.tex
\chapter{On shell renormalisation of the effective potential -- zero momentum approach}

%\section{Zero momentum approach}
%\subsubsection{No mass term}
We are concerned with the theory described by \ref{scalar_electrodynamics}:
\begin{align}
\mathcal{L} = -\frac{1}{4}F_{\mu\nu}F^{\mu\nu}+ 
\frac{1}{2}(\partial_\mu\varphi_1 - eA_\mu\varphi_2)^2\\
+\frac{1}{2}(\partial_\mu\varphi_2+eA_\mu\varphi_1)^2
-\cl\lambda(\varphi_1^2+\varphi_2^2)^2. \notag
\end{align}
Here, as well, we will consider potential up to order $e^4$. \\
We start with the 1-loop level potential with counterterms:
\begin{align}
\VRIloop = \frac{3e^4\varphi_1^4}{64\pi^2}\Big(-\frac{2}{\epsilon}+\gamma_E-\frac{5}{6}+
\log{\frac{e^2\varphi_1^2}{4\pi\mu^2}}\Big) + \cl\dl\varphi_1^4+\cm\dm\varphi_1^2
\end{align}
We will present three different versions of possible renormalisation conditions. 
\section{Vanishing derivatives}
As renormalisation conditions we impose that:
\begin{align}
\DphiII{}\VRIloop\Big|_{v} &= 0 \\
\DphiIV{}\VRIloop\Big|_{v} &= 0
\end{align}
Codesponding derivatives are:
\begin{align}
\DphiII{}\VRIloop = &\frac{9e^4\varphi_1^2}{16\pi^2}\Big(-\frac{2}{\epsilon}+\gamma_E-
\frac{5}{6}+\log{\frac{e^2\varphi_1^2}{4\pi\mu^2}}\Big)+\frac{21e^4}{32\pi^2}\varphi_1^2+ 
12\cl\dl\varphi_1^2+2\cm\dm \\
\DphiIV{}\VRIloop = &\frac{9e^4\varphi_1^2}{8\pi^2}\Big(-\frac{2}{\epsilon}+\gamma_E-
\frac{5}{6}+\log{\frac{e^2\varphi_1^2}{4\pi\mu^2}}\Big)+\frac{75e^4}{16\pi^2}+24\cl\dl.
\end{align}
So the conditions take form:
\begin{align}
&\frac{9e^4v^2}{16\pi^2}\Big(-\frac{2}{\epsilon}+\gamma_E-
\frac{5}{6}+\log{\frac{e^2v^2}{4\pi\mu^2}}\Big)+\frac{21e^4}{32\pi^2}v^2+ 
12\cl\dl v^2+2\cm\dm = 0\\
&\frac{9e^4v^2}{8\pi^2}\Big(-\frac{2}{\epsilon}+\gamma_E-
\frac{5}{6}+\log{\frac{e^2v^2}{4\pi\mu^2}}\Big)+\frac{75e^4}{16\pi^2}+24\cl\dl = 0.
\end{align}
Solving for $\dl$ and $\dm$ we have:
\begin{align}
\dl = &\frac{-e^4}{64\pi^2\cl}\Big(3\big(-\frac{2}{\epsilon}+\gamma_E-
\frac{5}{6}+\log\frac{e^2v^2}{4\pi\mu^2}\big)+\frac{25}{2}\Big) \\
\dm = &\frac{27e^4v^2}{32\pi^2\cm}.
\end{align}
Then, the renormalised potential is:
\begin{align}
V_R = \cl\lambda\varphi_1^4 + \frac{e^4\varphi_1^4}{64\pi^2}\Big(3\log\frac{\varphi_1^2}{v^2}- 
\frac{25}{2}\Big)+
\frac{27e^4v^2\varphi_1^2}{32\pi^2}
\end{align}
We would like to write $V_R$ in terms of $M_P$ -- the physical mass and $v$.
First relation is $\lambda = \frac{M_P^2}{12\cl v^2}$. Potential writen with this substitution 
becames:
\begin{align}
V_R = \frac{M_P^2\varphi_1^4}{12v^2} + \frac{e^4\varphi_1^4}{64\pi^2}\Big(3\log
\frac{\varphi_1^2}{v^2}- 
\frac{25}{2}\Big)+
\frac{27e^4v^2\varphi_1^2}{32\pi^2}
\end{align}
%Writing it with $M_P = 12\cl\lambda v^2$ -- physical mass, as a parameter, instead of 
%$\lambda$ gives us:
%\begin{align}
%\VRIloop = \frac{3e^4\varphi^4}{64\pi^2}\Big(-\frac{2}{\epsilon}+\gamma_E-\frac{5}{6}+
%\log{\frac{e^2\varphi_1^2}{4\pi\mu^2}}\Big) + \cl\dl\varphi_1^4+\cm\dm\varphi_1^2
%\end{align}
To write $e$ in terms of $M_P$ and $v$ we use the condition that 
\begin{align}
\DphiI{}\VR\Big|_{v} &= 0 
\end{align}
as $v$ is by definition minimum of the potential. \\
It gives the relation:
\begin{align}
\frac{M_P^2v}{3}+\frac{e^4v^3}{16\pi^2}\Big(-\frac{25}{2}\Big)+\frac{3e^4v^3}{32\pi^2}+
\frac{27e^4v^3}{16\pi^2} = 0
\end{align}
Thus, we conclude that: 
\begin{equation}
e^4=\frac{-M_P^2\pi^2}{3v^2}.
\end{equation}
This, unfortunatelly, is unacceptable, as then $e$ is no longer a real number, which is 
unphysical.
Thus, we conclude, that presented renormalisation method is not working and we need to search for 
another. One of possible ways is to expand the method to finite momentum. \\
\subsection{Comparison}
\todo{czy zostawić, to co poniżej}
We will now investigate, whether this has some chance of working by comparing 
above "zero momentum" method, only with first and second derivative, with "finite momentum" 
method and it's zero momentum limit. \\
With the conditions:
\begin{align}
\DphiI{}\VRIloop\Big|_{v} &= 0 \\
\DphiII{}\VRIloop\Big|_{v} &= 0
\end{align}
renormalisation constants $\dl$ and $\dm$ take form:
\begin{align}
\dl =& -\frac{e^4}{8\pi^2}-\frac{3e^4}{16\pi^2}\Big(-\frac{2}{\epsilon}-\gamma_E+\log{
\frac{e^2v^2}{4\pi\mu^2}}\Big)\\
\dm =& \frac{3e^4v^2}{16\pi^2}.
\end{align}
And the renormalised potential is equal to:
\begin{align}
V_R = \frac{M_P^2}{48\cl v^2}+\frac{e^4\varphi_1^4}{128\pi^2}\Big(6\log{
\frac{\varphi_1^2}{v^2}} - 9\Big)+\frac{3e^4v^2\varphi_1^2}{32\pi^2}.
\end{align}
Now, however, the condition for $v$ is meaningless ass we already used it in renormalisation 
conditions. Nethertheless, we want to investigate how values of $\dl$ and $\dm$ are compared 
to other approaches.
%\subsubsection{With mass term}
%For comparison, we include also a version of this aprouch steming from \ref{mass_term}.
%Inclusion of the mass term do not change the
%form of $\dl$ and $\dm$. The frist difference occurs in the potential. \\
%First we will describe the case with derivatives II and IV used in renormalisation conditions. 
%Then the potential is equal:
%\begin{align}
%V_R = \cl\lambda\varphi_1^4 + \cm m^2\varphi_1^2 +
%\frac{e^4\varphi_1^4}{64\pi^2}\Big(3\log\frac{\varphi_1^2}{v^2}- 
%\frac{25}{2}\Big)+
%\frac{27e^4v^2\varphi_1^2}{32\pi^2}.
%\end{align}
%Now, we have that $M_P = 12\cl\lambda v^2 + 2\cm m^2$. 
%No, we will use the condition, that:
%\begin{align}\label{tree_first}
%\DphiI{}\VT\Big|_v = 0
%\end{align}
%we have that $4\cl\lambda v^3+2\cm m^2v = 0$, so $\lambda = -\frac{\cm m^2}{2\cl v^2}$, so 
%\begin{align}
%m^2&=\frac{-M_P^2}{4\cm} \textrm{\ \ and}\\ 
%\lambda &= \frac{M_P^2}{8\cl v^2}. 
%\end{align}
%Writing $V_R$ with respect to that gives:
%\todo{pytanie}
%\begin{align}
%V_R = \frac{M_P^2\varphi_1^4}{8v^2}-\frac{M_P^2\varphi_1^2}{4} + 
%\frac{e^4\varphi_1^4}{64\pi^2}\Big(3\log
%\frac{\varphi_1^2}{v^2}- 
%\frac{25}{2}\Big)+
%\frac{27e^4v^2\varphi_1^2}{32\pi^2}
%\end{align}
%From this we have, that:
%\begin{equation}
%\frac{e^4v^3}{\pi^2} = 0,
%\end{equation}
%which is also a not safisying result. \\
%However, if we drop the condition, that 
%$\DphiI{}\VT\Big|_v = 0$, we have potential in the form:
%\begin{align}
%V_R = \frac{M_P^2-2\cm m^2}{12v^2}\varphi_1^4+\cm m^2\varphi_1^2 + 
%\frac{e^4\varphi_1^4}{64\pi^2}\Big(3\log
%\frac{\varphi_1^2}{v^2}- 
%\frac{25}{2}\Big)+
%\frac{27e^4v^2\varphi_1^2}{32\pi^2}
%\end{align}
%From thism, using the condition that $\DphiI{}\VR\Big|_v = 0$, we can derive the 
%correspondence between $e$ and $M_P$, $v$ and $m$:
%\begin{equation}
%e^4 = -\frac{(M_P^2+4\cm m^2)\pi^2}{3v^2},
%\end{equation}
%which is finally a sensible result as it can be realised with real, positive $e$. However, then 
%it must hold that $m^2<-\frac{M_P^2}{4\cm}$.
%\subsection{"Limits" version}


\section{Half \MSbar-Half Onshell scheme}
Due to (so far) lack of experimenttal data of coupling $\lambda$ in considered theories, 
the on shell condition for that constatnt renders itself meaningless. \\
Thus, we propose mixed scheme, where we demand that the physical mass remain unchanged due to 
the one-loop corrections, but for the coupling case, we demand only that the $\dl$ 
counterterm is such that the fourth  derivative of the renormalzsed effective potential is finite 
-- in the \MSbar manner.
%\section{Finite momentum}

%\section{Zero momentum}
Here we impose following renormalisation conditions:
\begin{align}
\DphiII{}\VRIloop\Big|_{v} &= 0 \\
\DphiIV{}\VRIloop\Big|_{v} &= \frac{9e^4}{8\pi^2}\Big(-\frac{5}{6}+\log\frac{e^2
v^2}{\mu^2}\Big)
+\frac{75e^4}{16\pi^2}
\end{align}
Written in the full form these conditions take form:
\begin{align}
&\frac{9e^4}{16\pi^2}\Big(-\frac{2}{\epsilon}+\gamma_E-\frac{5}{6}+\log\frac{e^2
v^2}{4\pi\mu^2}\Big)v^2+ \notag\\
& 12\cl\dl v^2+\frac{21e^4}{32\pi^2}v^2+2\cm\dm = 0 \\
&\frac{9e^4}{8\pi^2}\Big(-\frac{2}{\epsilon}+\gamma_E-\frac{5}{6}+\log\frac{e^2v^2}{4\pi\mu^2}\Big)+
24\cl\dl+\frac{75e^4}{16\pi^2} = \notag \\
&\frac{9e^4}{8\pi^2}\Big(-\frac{5}{6}+\log\frac{e^2v^2}{\mu^2}\Big)+\frac{75e^4}{16\pi^2}
\end{align}
After solving equations for $\dm$ and $\dl$ we obtain:
\begin{align}
\dm &= -\frac{3e^4v^2}{32\cm\pi^2}\Big(1+3
\log\frac{e^2v^2}{\mu^2}\Big) \\
\dl &= -\frac{3e^4}{64\cl\pi^2}\Big(-\frac{2}{\epsilon}+\gamma_E-\log(4\pi)\Big)
\end{align}
The renormalized potential is then:
\begin{align}
\VR = &\cl\lambda\varphi_1^4 + \notag \\ 
&\frac{3e^4}{64\pi^2}\Big(-\frac{5}{6}+\log\frac{e^2\varphi_1^2}{\mu^2}\Big)\varphi_1^4+ \notag\\
&\frac{3e^4v^2}{32\pi^2}\Big(-1-3\log\frac{e^2v^2}{\mu^2}\Big)\varphi_1^2
\end{align}
From the tree level potential we have the relation $\lambda = \frac{M_P^2}{12\cl v^2}$. 
Written in these terms we have:
\begin{align}
\VR = &\frac{M_P^2}{12v^2}\varphi_1^4 + \notag \\ 
&\frac{3e^4}{64\pi^2}\Big(-\frac{5}{6}+\log\frac{e^2\varphi_1^2}{\mu^2}\Big)\varphi_1^4+ \notag\\
&\frac{3e^4v^2}{32\pi^2}\Big(-1-3\log\frac{e^2v^2}{\mu^2}\Big)\varphi_1^2
\end{align}
We can bind $e$ to $M_P$ and $v$ at the loop level, from the definition of VEV:
\begin{align}
\DphiI\VR \Big|_v= 0
\end{align}
This gives the condition:
\begin{align}
\frac{M_P^2v}{3}-\frac{e^4v^3}{4\pi^2}-\frac{3e^4v^3}{8\pi^2}\log\frac{e^2v^2}{\mu^2}=0
\end{align}
Setting scale parameter $\mu$ to the effective mass of the vector, namely $ev$, we have 
simpler form of:
\begin{align}
-\frac{e^4v^3}{4\pi^2}+\frac{M_P^2v}{3}=0
\end{align}
Which gives:
\begin{align}
e^4 = \frac{4M_P^2\pi^2}{3v^2}
\end{align}
Writing potential with this substitutions yields asdad:
\begin{align}
V_R = \frac{M_P^2}{16v^2}\Big(\frac{1}{2}+\log\frac{\varphi_1^2}{v^2}\Big)\varphi_1^4- 
\frac{M_P^2}{8}\varphi_1^2
\end{align}
Potential can be written also in terms of $v$ and $e$:
\begin{equation}
V_R=\frac{3e^4}{64\pi^2}\Big(\frac{1}{2}+\log\frac{\varphi_1^2}{v^2}\Big)\varphi_1^4- 
\frac{3e^2v^2}{32\pi^2}\varphi_1^2
\end{equation}
We can also derive direct relation between $\lambda$ and $e^4$, namely:
\begin{align}
\lambda = \frac{e^4}{16\cl\pi^2},
\end{align}
and the ratio of masses of scalar and vector:
\begin{align}
\frac{M_P^2}{m(V)^2} = \frac{\frac{3e^4v^2}{4\pi^2}}{e^2v^2} = \frac{3e^2}{4\pi^2}
\end{align}
\section{On shell with $\varphi^4$ potential}\label{onshellphi4}
Although the previous renormalisation succeeded in making the potential finite and being in 
agreement with radiative symmetry breaking it has one last problem -- the resulting potential have 
square term in the field. \\
We would like to investigate, whether one can renormalise theory such that tree 
level mass is the phsical mass, namely $\DphiII{}\VR\Big|_{v}$ (on-shell condition) and at the 
same time square term vanishes. \\
We will follow, what was done in \cite{Coleman1973}, as described in 
\ref{CW_renormalisation_scalar_el} and by hand put the condition to square terms to vanish, 
which is equvalent to the condition for $\DphiII{\VR}\Dat{0} = 0$. \\
We can see, that in the renormalised potential:
\begin{align}
\VR = \cl\lambda\varphi_1^4 + \frac{3e^4\varphi_1^4}{64\pi^2}\Big(-\frac{2}{\epsilon}+\gamma_E- 
\frac{5}{6}+
\log{\frac{e^2\varphi_1^2}{4\pi\mu^2}}\Big) + \cl\dl\varphi_1^4+\cm\dm\varphi_1^2
\end{align}
only term square in the fields is $\cm\dm\varphi_1^2$, therefore $\dm$ should be zero. \\
Note, that it is not the same as disregarding $\dm$ automaticly. As stated in \cite{Coleman1973}, 
the theory has no a priori symmetry for $\dm$ to be $0$ and we are respectful to that. 
It just so happens that in our regularisation scheme, if we want to have no square terms in the 
resulting potential (or to $\DphiII{\VR}\Dat{0}$ to vanish, which is equivalent), we need to put 
$\dm = 0$. With different regularisation to satisfy this condition, we would have different $\dm$, 
as seen in \cite{Coleman1973}, written here at \ref{CW_B_dm}. \\
%Nevertheless, we are reluctant, to phrase the con
Therefore we impose following renormalising conditions:
\begin{align}
\DphiII{}\VRIloop\Big|_{v}=0 \\
%\dm = 0
\DphiII{}\VR\Dat{0}=0
\end{align}
Written in the full form these conditions take form:
\begin{align}
&\frac{9e^4}{16\pi^2}\Big(-\frac{2}{\epsilon}+\gamma_E-\frac{5}{6}+\log\frac{e^2
v^2}{4\pi\mu^2}\Big)v^2+ \notag\\
& 12\cl\dl v^2+\frac{21e^4}{32\pi^2}v^2+2\cm\dm = 0 \\
%&\dm = 0
&\DphiII{}\VR\Dat{0} = 0
\end{align}
After solving equations for $\dm$ and $\dl$ we obtain:
\begin{align}
\dm &= 0 \\
\dl &= -\frac{3e^4}{64\cl\pi^2}\Big(-\frac{2}{\epsilon}+\gamma_E-\frac{5}{6}+
\log\frac{e^2v^2}{4\pi\mu^2}\Big)-\frac{7e^4}{128\cl\pi^2}
\end{align}
The renormalized potential is then:
\begin{align}
\VR = &\cl\lambda\varphi_1^4 +
\frac{3e^4}{64\pi^2}\Big(-\frac{7}{6}+\log\frac{\varphi_1^2}{v^2}\Big)\varphi_1^4
\end{align}
From the tree level potential we have the relation $\lambda = \frac{M_P^2}{12\cl v^2}$. 
Written in these terms we have:
\begin{align}
\VR = &\frac{M_P^2}{12v^2}\varphi_1^4 +
\frac{3e^4}{64\pi^2}\Big(-\frac{7}{6}+\log\frac{\varphi_1^2}{v^2}\Big)\varphi_1^4
\end{align}
We can bind $e$ to $M_P$ and $v$ at the loop level, from the definition of VEV:
\begin{align}
\DphiI\VR \Big|_v= 0
\end{align}
This gives the condition:
\begin{align}
\frac{M_P^2v}{3}+\frac{3e^4v^3}{16\pi^2}\Big(-\frac{7}{6}\Big)+\frac{3e^4v^3}{32\pi^2}=0
\end{align}
Which gives:
\begin{align}
e^4 = \frac{8M_P^2\pi^2}{3v^2}
\end{align}
Writing potential with this substitutions yields asdad:
\begin{align}
V_R = \frac{M_P^2}{8v^2}\Big(-\frac{1}{2}+\log\frac{\varphi_1^2}{v^2}\Big)\varphi_1^4
\end{align}
Potential can be written also in terms of $v$ and $e$:
\begin{equation}
V_R=\frac{3e^4}{64\pi^2}\Big(-\frac{1}{2}+\log\frac{\varphi_1^2}{v^2}\Big)\varphi_1^4
\end{equation}
Note, that, this is exactly same potential as in \cite{Coleman1973} and \ref{MSbar_result}, 
so, following quantities must be the same as there: \\
\hspace*{0.5cm}- relation between $\lambda$ and $e^4$:
\begin{align}
\lambda = \frac{e^4}{32\cl\pi^2},
\end{align}
\hspace*{0.5cm}- ratio between masses of scalar and vector:
\begin{align}
\frac{M_P^2}{m(V)^2} = \frac{\frac{3e^4v^2}{8\pi^2}}{e^2v^2} = \frac{3e^2}{8\pi^2}
\end{align}



