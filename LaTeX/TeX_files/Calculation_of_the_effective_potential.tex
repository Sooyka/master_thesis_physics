% mainfile: ../master_thesis_GW.tex
\chapter{Introduction -- calculation of the unrenormalized effective potential}

Tree level potential in our model is:
\begin{equation}
V_T = \cl\lambda(\varphi_1^2+\varphi_2^2)^2.
\end{equation}


The one loop correction to the effective potential is calculated as a sum of the following diagrams:
\\
\begin{tikzpicture}[line width=1.5 pt, scale=1.3]
    \begin{scope}
%	    \draw[scalar] (0:1)--(0,0);
        \draw[scalar] (0,-0.5) --(1.5,-0.5);
	    \draw[scalar] (1,0) arc (180:-180:.5);
	    %\draw[vector] (2,0) arc (0:-180:.5);
	    \draw[scalar] (1.5,-0.5) --(3,-0.5);
	    \node at (-.2,-0.5) {$\varphi_1$};
	    \node at (3.2,-0.5) {$\varphi_1$};
%	    \node at (40:2) {$\varphi_1$};
        \node at (1.5,0.8) {$\varphi_1$};
    \end{scope},
%\end{tikzpicture}
%\begin{tikzpicture}[line width=1.5 pt, scale=1.3]
    \node at (3.6, -0.5) {$\textrm{, }$};
    \begin{scope}[shift = {(3.5,0)}]
    %	\draw[scalar] (0:1)--(0,0);
        \draw[scalar] (1,1) -- (1.5,0.5); 
    %    \draw[scalar] (1.5,0.5) -- (1,1); 
        \draw[scalar] (1.5,0.5) -- (2,1); 
	    \draw[scalar] (1,0) arc (180:-180:.5);
    %	\draw[vector] (2,0) arc (0:-180:.5);
        \draw[scalar] (1,-1)--(1.5,-0.5);
	    \draw[scalar] (1.5,-0.5)--(2,-1);
	    \node at (0.8,-1.2) {$\varphi_1$};
	    \node at (2.2,-1.2) {$\varphi_1$};
	    \node at (0.8,1.2) {$\varphi_1$};
	    \node at (2.2,1.2) {$\varphi_1$};
	    \node at (0.75,0) {$\varphi_1$};
    %	\node at (40:2) {$\varphi_1$};
        
    \end{scope}
    \node at (6.2, -0.5) {$\textrm{, }$};
    \begin{scope}[shift = {(6.5,0)}]
    %	\draw[scalar] (0:1)--(0,0);
        \draw[scalar] (1,1) -- (1.5,0.5); 
    %    \draw[scalar] (1.5,0.5) -- (1,1); 
        \draw[scalar] (1.5,0.5) -- (2,1); 
	    \draw[scalar] (1,0) arc (180:-180:.5);
    %	\draw[vector] (2,0) arc (0:-180:.5);
%        \draw[scalar] (1,-1)--(1.5,-0.5);
	    \draw[scalar] (1.9,-0.3)--(2.7,-0.5);
	    \draw[scalar] (2.2, -1.1)--(1.9,-0.3);
	    \node at (2.4,-1.3) {$\varphi_1$};
	    \node at (2.9,-0.6) {$\varphi_1$};
	    \node at (0.8,1.2) {$\varphi_1$};
	    \node at (2.2,1.2) {$\varphi_1$};
	    \node at (0.75,0.3) {$\varphi_1$};
    %	\node at (40:2) {$\varphi_1$};
        \draw[scalar] (1.1,-0.3)--(0.3,-0.5);
	    \draw[scalar] (0.8, -1.1)--(1.1,-0.3);
	    \node at (0.6,-1.3) {$\varphi_1$};
	    \node at (0.1,-0.6) {$\varphi_1$};
        
    \end{scope}
    \node at (10, -0.5) {$\textrm{, \ .\ .\ .\ ,}$};
    
\end{tikzpicture} \\

\begin{tikzpicture}[line width=1.5 pt, scale=1.3]
    \begin{scope}
%	    \draw[scalar] (0:1)--(0,0);
        \draw[scalar] (0,-0.5) --(1.5,-0.5);
	    \draw[scalar] (1,0) arc (180:-180:.5);
	    %\draw[vector] (2,0) arc (0:-180:.5);
	    \draw[scalar] (1.5,-0.5) --(3,-0.5);
	    \node at (-.2,-0.5) {$\varphi_1$};
	    \node at (3.2,-0.5) {$\varphi_1$};
%	    \node at (40:2) {$\varphi_1$};
        \node at (1.5,0.8) {$\varphi_2$};
    \end{scope},
%\end{tikzpicture}
%\begin{tikzpicture}[line width=1.5 pt, scale=1.3]
    \node at (3.6, -0.5) {$\textrm{, }$};
    \begin{scope}[shift = {(3.5,0)}]
    %	\draw[scalar] (0:1)--(0,0);
        \draw[scalar] (1,1) -- (1.5,0.5); 
    %    \draw[scalar] (1.5,0.5) -- (1,1); 
        \draw[scalar] (1.5,0.5) -- (2,1); 
	    \draw[scalar] (1,0) arc (180:-180:.5);
    %	\draw[vector] (2,0) arc (0:-180:.5);
        \draw[scalar] (1,-1)--(1.5,-0.5);
	    \draw[scalar] (1.5,-0.5)--(2,-1);
	    \node at (0.8,-1.2) {$\varphi_1$};
	    \node at (2.2,-1.2) {$\varphi_1$};
	    \node at (0.8,1.2) {$\varphi_1$};
	    \node at (2.2,1.2) {$\varphi_1$};
	    \node at (0.75,0) {$\varphi_2$};
    %	\node at (40:2) {$\varphi_1$};
        
    \end{scope}
    \node at (6.2, -0.5) {$\textrm{, }$};
    \begin{scope}[shift = {(6.5,0)}]
    %	\draw[scalar] (0:1)--(0,0);
        \draw[scalar] (1,1) -- (1.5,0.5); 
    %    \draw[scalar] (1.5,0.5) -- (1,1); 
        \draw[scalar] (1.5,0.5) -- (2,1); 
	    \draw[scalar] (1,0) arc (180:-180:.5);
    %	\draw[vector] (2,0) arc (0:-180:.5);
%        \draw[scalar] (1,-1)--(1.5,-0.5);
	    \draw[scalar] (1.9,-0.3)--(2.7,-0.5);
	    \draw[scalar] (2.2, -1.1)--(1.9,-0.3);
	    \node at (2.4,-1.3) {$\varphi_1$};
	    \node at (2.9,-0.6) {$\varphi_1$};
	    \node at (0.8,1.2) {$\varphi_1$};
	    \node at (2.2,1.2) {$\varphi_1$};
	    \node at (0.75,0.3) {$\varphi_2$};
    %	\node at (40:2) {$\varphi_1$};
        \draw[scalar] (1.1,-0.3)--(0.3,-0.5);
	    \draw[scalar] (0.8, -1.1)--(1.1,-0.3);
	    \node at (0.6,-1.3) {$\varphi_1$};
	    \node at (0.1,-0.6) {$\varphi_1$};
        
    \end{scope}
    \node at (10, -0.5) {$\textrm{, \ .\ .\ .\ ,}$};
    
\end{tikzpicture} \\

\begin{tikzpicture}[line width=1.5 pt, scale=1.3]
    \begin{scope}
%	    \draw[scalar] (0:1)--(0,0);
        \draw[scalar] (0,-0.5) --(1.5,-0.5);
	    \draw[vector] (1,0) arc (180:-180:.5);
	    %\draw[vector] (2,0) arc (0:-180:.5);
	    \draw[scalar] (1.5,-0.5) --(3,-0.5);
	    \node at (-.2,-0.5) {$\varphi_1$};
	    \node at (3.2,-0.5) {$\varphi_1$};
%	    \node at (40:2) {$\varphi_1$};
        \node at (1.5,0.8) {$\gamma$};
    \end{scope},
%\end{tikzpicture}
%\begin{tikzpicture}[line width=1.5 pt, scale=1.3]
    \node at (3.6, -0.5) {$\textrm{, }$};
    \begin{scope}[shift = {(3.5,0)}]
    %	\draw[scalar] (0:1)--(0,0);
        \draw[scalar] (1,1) -- (1.5,0.5); 
    %    \draw[scalar] (1.5,0.5) -- (1,1); 
        \draw[scalar] (1.5,0.5) -- (2,1); 
	    \draw[vector] (1,0) arc (180:-180:.5);
    %	\draw[vector] (2,0) arc (0:-180:.5);
        \draw[scalar] (1,-1)--(1.5,-0.5);
	    \draw[scalar] (1.5,-0.5)--(2,-1);
	    \node at (0.8,-1.2) {$\varphi_1$};
	    \node at (2.2,-1.2) {$\varphi_1$};
	    \node at (0.8,1.2) {$\varphi_1$};
	    \node at (2.2,1.2) {$\varphi_1$};
	    \node at (0.75,0) {$\gamma$};
    %	\node at (40:2) {$\varphi_1$};
        
    \end{scope}
    \node at (6.2, -0.5) {$\textrm{, }$};
    \begin{scope}[shift = {(6.5,0)}]
    %	\draw[scalar] (0:1)--(0,0);
        \draw[scalar] (1,1) -- (1.5,0.5); 
    %    \draw[scalar] (1.5,0.5) -- (1,1); 
        \draw[scalar] (1.5,0.5) -- (2,1); 
	    \draw[vector] (1,0) arc (180:-180:.5);
    %	\draw[vector] (2,0) arc (0:-180:.5);
%        \draw[scalar] (1,-1)--(1.5,-0.5);
	    \draw[scalar] (1.9,-0.3)--(2.7,-0.5);
	    \draw[scalar] (2.2, -1.1)--(1.9,-0.3);
	    \node at (2.4,-1.3) {$\varphi_1$};
	    \node at (2.9,-0.6) {$\varphi_1$};
	    \node at (0.8,1.2) {$\varphi_1$};
	    \node at (2.2,1.2) {$\varphi_1$};
	    \node at (0.75,0.3) {$\gamma$};
    %	\node at (40:2) {$\varphi_1$};
        \draw[scalar] (1.1,-0.3)--(0.3,-0.5);
	    \draw[scalar] (0.8, -1.1)--(1.1,-0.3);
	    \node at (0.6,-1.3) {$\varphi_1$};
	    \node at (0.1,-0.6) {$\varphi_1$};
        
    \end{scope}
    \node at (10, -0.5) {$\textrm{, \ .\ .\ ..}$};
    
\end{tikzpicture}

where: \\
%\hspace*{2cm}
\begin{tikzpicture}[line width=1.5 pt, scale=1.3]
\begin{scope}[shift = {(6.5,0)}]
    %	\draw[scalar] (0:1)--(0,0);
        \draw[scalar] (1,1) -- (1.5,0.5); 
    %    \draw[scalar] (1.5,0.5) -- (1,1); 
        \draw[scalar] (1.5,0.5) -- (2,1); 
	    \draw[scalar] (1,0) arc (180:-180:.5);
    %	\draw[vector] (2,0) arc (0:-180:.5);
%        \draw[scalar] (1,-1)--(1.5,-0.5);
	    \draw[scalar] (1.9,-0.3)--(2.7,-0.5);
	    \draw[scalar] (2.2, -1.1)--(1.9,-0.3);
	    \node at (2.4,-1.3) {$\varphi_1$};
	    \node at (2.9,-0.6) {$\varphi_1$};
	    \node at (0.8,1.2) {$\varphi_1$};
	    \node at (2.2,1.2) {$\varphi_1$};
	    \node at (0.75,0.3) {$\varphi_1$};
    %	\node at (40:2) {$\varphi_1$};
        \draw[scalar] (1.1,-0.3)--(0.3,-0.5);
	    \draw[scalar] (0.8, -1.1)--(1.1,-0.3);
	    \node at (0.6,-1.3) {$\varphi_1$};
	    \node at (0.1,-0.6) {$\varphi_1$};
        \node at (2.7,0.5) {... $n$ pairs};
        \node at (3.2,0.1) {of $\varphi_1$ legs total};
    \end{scope}
\end{tikzpicture}

\vspace*{-3cm}\begin{align}
\hspace*{3cm}=i\int \frac{\textrm{d}^4k}{(2\pi)^4}\frac{1}{2n}\left(\frac{12\cl\lambda
\varphi_1^2}
{k^2+i\varepsilon}\right)^n 
\end{align} \\[8pt]

\begin{tikzpicture}[line width=1.5 pt, scale=1.3]
\begin{scope}[shift = {(6.5,0)}]
    %	\draw[scalar] (0:1)--(0,0);
        \draw[scalar] (1,1) -- (1.5,0.5); 
    %    \draw[scalar] (1.5,0.5) -- (1,1); 
        \draw[scalar] (1.5,0.5) -- (2,1); 
	    \draw[scalar] (1,0) arc (180:-180:.5);
    %	\draw[vector] (2,0) arc (0:-180:.5);
%        \draw[scalar] (1,-1)--(1.5,-0.5);
	    \draw[scalar] (1.9,-0.3)--(2.7,-0.5);
	    \draw[scalar] (2.2, -1.1)--(1.9,-0.3);
	    \node at (2.4,-1.3) {$\varphi_1$};
	    \node at (2.9,-0.6) {$\varphi_1$};
	    \node at (0.8,1.2) {$\varphi_1$};
	    \node at (2.2,1.2) {$\varphi_1$};
	    \node at (0.75,0.3) {$\varphi_2$};
    %	\node at (40:2) {$\varphi_1$};
        \draw[scalar] (1.1,-0.3)--(0.3,-0.5);
	    \draw[scalar] (0.8, -1.1)--(1.1,-0.3);
	    \node at (0.6,-1.3) {$\varphi_1$};
	    \node at (0.1,-0.6) {$\varphi_1$};
        \node at (2.7,0.5) {... $n$ pairs};
        \node at (3.2,0.1) {of $\varphi_1$ legs total};
    \end{scope}
\end{tikzpicture}

\vspace*{-3cm}\begin{align}
\hspace*{3cm}=i\int \frac{\textrm{d}^4k}{(2\pi)^4}\frac{1}{2n}&\left(
\frac{1}{3}\frac{12\cl\lambda\varphi_1^2}
{k^2+i\varepsilon}\right)^n 
\end{align} \\[8pt]

\begin{tikzpicture}[line width=1.5 pt, scale=1.3]
\begin{scope}[shift = {(6.5,0)}]
    %	\draw[scalar] (0:1)--(0,0);
        \draw[scalar] (1,1) -- (1.5,0.5); 
    %    \draw[scalar] (1.5,0.5) -- (1,1); 
        \draw[scalar] (1.5,0.5) -- (2,1); 
	    \draw[vector] (1,0) arc (180:-180:.5);
    %	\draw[vector] (2,0) arc (0:-180:.5);
%        \draw[scalar] (1,-1)--(1.5,-0.5);
	    \draw[scalar] (1.9,-0.3)--(2.7,-0.5);
	    \draw[scalar] (2.2, -1.1)--(1.9,-0.3);
	    \node at (2.4,-1.3) {$\varphi_1$};
	    \node at (2.9,-0.6) {$\varphi_1$};
	    \node at (0.8,1.2) {$\varphi_1$};
	    \node at (2.2,1.2) {$\varphi_1$};
	    \node at (0.75,0.3) {$\gamma$};
    %	\node at (40:2) {$\varphi_1$};
        \draw[scalar] (1.1,-0.3)--(0.3,-0.5);
	    \draw[scalar] (0.8, -1.1)--(1.1,-0.3);
	    \node at (0.6,-1.3) {$\varphi_1$};
	    \node at (0.1,-0.6) {$\varphi_1$};
        \node at (2.7,0.5) {... $n$ pairs};
        \node at (3.2,0.1) {of $\varphi_1$ legs total};
    \end{scope}
\end{tikzpicture}

\vspace*{-3cm}\begin{align}
\hspace*{5cm}=i\int \frac{\textrm{d}^4k}{(2\pi)^4}\frac{1}{2n}&\left(\frac{2e^2\frac{1}{2}\varphi_1^2}
{k^2+i\varepsilon}\right)^n(g^\mu_{\ \,\mu} - 1).
\end{align} \\[8pt]

Summing all the diagrams in series gives:


\begin{align}
V_{1L} = i\int \frac{\textrm{d}^4k}{(2\pi)^4}\sum\limits_{n=1}^{\infty}
\frac{1}{2n}&\left(\frac{12\cl\lambda\varphi_1^2}
{k^2+i\varepsilon}\right)^n + \\
i\int \frac{\textrm{d}^4k}{(2\pi)^4}\sum\limits_{n=1}^{\infty}\frac{1}{2n}&
\left(\frac{1}{3}\frac{12\cl\lambda\varphi_1^2}
{k^2+i\varepsilon}\right)^n +\\
i\int \frac{\textrm{d}^4k}{(2\pi)^4}\sum\limits_{n=1}^{\infty}\frac{1}{2n}&
\left(\frac{2e^2\frac{1}{2}\varphi_1^2}
{k^2+i\varepsilon}\right)^n(g^\mu_{\ \,\mu} - 1).
\end{align}\label{divergent sums}

These integrals seems to be hideously infrared divergent. 
We can, however Wick rotate them to the Euklidean space and perform summation to get:
\begin{align}\label{eff_before_reg}
V_{1L}=&\frac{1}{2} \int\frac{d^4k}{(2\pi)^4}\ln{\left(1+\frac{12\cl\lambda\varphi_1^2}{k^2}
\right)} +\\
&\frac{1}{2} \int\frac{d^4k}{(2\pi)^4}\ln{\left(1+\frac{4\cl\lambda\varphi_1^2}{k^2}
\right)} +\\ 
&\frac{3}{2} \int\frac{d^4k}{(2\pi)^4}\ln{\left(1+\frac{e^2\varphi_1^2}{k^2}\right)}
\end{align}
Now, integrals have only singularity at $\varphi_1 = 0$. \\
There are now several ways to perform these integrals. We will present here briefly two of them: 
cut-off and
dimentional regularisation (DR). \\
For all of our further calculations in the thesis, we will use DR, 
but cut-off will apear in \ref{CWchapter} 
as a method used in \cite{Coleman1973}. \\
\section{Cut-off}
The idea of this method is to restrict the integration to $k^2 \leq \Lambda$, for a 
parameter $\Lambda$. \\
Integrating \ref{eff_before_reg} up to $\Lambda$ gives:
\begin{align}
V_{1L}=
&\frac{3\cl\lambda\Lambda^2}{8\pi^2}\varphi_c^2+\frac{9\cl\lambda^2\varphi_c^4}{4\pi^2}
\left(\ln\frac{12\cl\lambda\varphi_c^2}{\Lambda^2}-\frac{1}{2}\right)  \notag \\
+&\frac{\cl\lambda\Lambda^2}{8\pi^2}\varphi_c^2+\frac{\cl\lambda^2\varphi_c^4}{4\pi^2}
\left(\ln\frac{4\cl\lambda\varphi_c^2}{\Lambda^2}-\frac{1}{2}\right)  \notag \\
+&\frac{3e^2\Lambda^2}{32\pi^2}\varphi_c^2+\frac{3e^4\varphi_c^4}{64\pi^2}
\left(\ln\frac{e^2\varphi_c^2}{\Lambda^2}-\frac{1}{2}\right).
\end{align}

\section{Dimentional regularisation}
The idea of this method is to observe, that the divergence is only in four dimentions 
and if we formally change expresions to $4-2\epsilon$ dimentions. 
It is performed by treating the integral as a linear functional depending on two parameters --
function and dimention given by some specific formula. Then we can calculate it with our 
function of interest and $D = 4-2\epsilon$ as arguments.  \\
%$D = 4-2\epsilon$
After passing to $D=4-2\epsilon$ dimensions the integrals take form: 
\begin{align}
V_{1L}=&\frac{\mu^\epsilon}{2} \int\frac{d^Dk}{(2\pi)^D}\ln{\left(1+\frac{12\cl\lambda\varphi_1^2}{k^2}
\right)}\\
+&\frac{\mu^\epsilon}{2} \int\frac{d^Dk}{(2\pi)^D}\ln{\left(1+\frac{4\cl\lambda\varphi_1^2}{k^2}
\right)} \\ 
+&\frac{(D-1)\mu^\epsilon}{2} \int\frac{d^Dk}{(2\pi)^D}\ln{\left(1+\frac{e^2\varphi_1^2}{k^2}\right)}.
\end{align}
This introduces the parameter $\mu$, \smalltodo{more about $\mu$}. \\
After calculating the integrals in $D$ dimentions we have:
\begin{align}
V_{1L}=&\frac{1}{4}\frac{(12\cl\lambda\varphi_1^2)^2}{(4\pi)^2}\left(-\frac{2}{\epsilon}
+\gamma_E-\frac{3}{2}+\log{\frac{(12\cl\lambda\varphi_1^2)^2}{4\pi\mu^2}}\right)  \\
+&\frac{1}{4}\frac{(4\cl\lambda\varphi_1^2)^2}{(4\pi)^2}\left(-\frac{2}{\epsilon}
+\gamma_E-\frac{3}{2}+\log{\frac{(4\cl\lambda\varphi_1^2)^2}{4\pi\mu^2}}\right)  \\
+&\frac{1}{4}\frac{3(e^2\varphi_1^2)^2}{(4\pi)^2}\left(-\frac{2}{\epsilon}+\gamma_E
-\frac{5}{6}+\log{\frac{e^2\varphi^2}{4\pi\mu^2}}\right).
\end{align}
\section{To renormalisation}
The task that is still left, is to mange the infinities that come when we pass to the limit 
and:
\begin{itemize} 
\item take $\Lambda \to \infty$, allowing for arbitrary large $k^2$ in the case of the cut-off
\item take $\epsilon \to 0$ and returning to $4$ dimentions in the case of DR.

\end{itemize}
This is the scope of renormalisation and is the main interest of this thesis. 
Different renormalisation techniques should ... \todo{how this is exactly}

It is the goal of this thesis to show wether two of them -- \MSbar and On shell give same results.











