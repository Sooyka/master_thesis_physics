% mainfile: ../master_thesis_GW.tex
\chapter{How did Coleman Weinberg do it?}
\section{$\lambda\varphi^4$ theory}
Coleman and Weinberg in \cite{Coleman1973} start with "$\lambda\varphi^4$" model, namely, with 
Lagransian:
\begin{align}
\mathcal{L} = \frac{1}{2}(\partial_\mu\varphi)^2-\frac{\lambda}{4!}\varphi^4+\frac{1}{2}
A(\partial_\mu\varphi)^2-\frac{1}{2}B\varphi^2-\frac{1}{4!}C\varphi^4,
\end{align}
where $A$, $B$, $C$ are renormalisation constants. \\
They then proceeed to calculations of the renormalised effective potential in this theory. \\
Tree level potential is
\begin{align}
V=\frac{\lambda}{4!}\varphi_c^4,
\end{align}
and up to the one loop level is:
\begin{align}
V = \frac{\lambda}{4!}\varphi_c^4+\frac{1}{2}B\varphi_c^2+\frac{1}{4!}C\varphi_c^4 + 
i\int\frac{d^4k}{(2\pi)^4}\sum\limits_{n=1}^\infty\frac{1}{2n}
\left(\frac{\frac{1}{2}\lambda\varphi_c^2}{k^2+i\epsilon}\right)^n.
\end{align} 
This expression seems "hideously infrared divergent". It is thus transformed into one 
in Euclidean space with apparent infrared divergence turned into logarithmic singularity at 
$\varphi_c=0$:
\begin{align}
V=\frac{\lambda}{4!}\varphi_c^4+\frac{1}{2}B\varphi_c^2+\frac{1}{4!}C\varphi_c^4 +
\frac{1}{2}\int\frac{d^4k}{(2\pi)^4}\ln\left(1+\frac{\lambda\varphi_c^2}{2k^2}\right).
\end{align}
This is then calculated using cut-off method, with cut off ad $k^2 = \Lambda^2$. 
The result is:
\begin{align}
V=\frac{\lambda}{4!}\varphi_c^4+\frac{1}{2}B\varphi_c^2+\frac{1}{4!}C\varphi_c^4+
\frac{\lambda\Lambda^2}{64\pi^2}\varphi_c^2+\frac{\lambda^2\varphi_c^4}{256\pi^2}
\left(\ln\frac{\lambda\varphi_c^2}{2\Lambda^2}-\frac{1}{2}\right).
\end{align}
\subsection{Renormalisation}
Then, they proceed, to renormalise the potential. First renormalisation condition is for the 
renormalised mass to vanish (eq. (3.5) in \cite{Coleman1973}). \\
They write is as:
\begin{align}
\frac{d^2V}{d\varphi_c^2}\Big|_0=0,
\end{align}
which can be suspitious, as, according to discusion in \cite{Coleman1973} rigth after equation 
(2.12b), we have that mass is equal to $\frac{d^2V}{d\varphi_c^2}\Big|_{\langle\varphi\rangle}$, 
where $\langle\varphi\rangle$ is VEV of $\varphi$. \\
A priori, it is not guaranteed in our theory, that symmetry is unbroken and 
$\langle\varphi\rangle = 0$. However, it is also not explicitly broken yet, so we left it for now. 
\\
This condition give us, that 
\begin{align}
B = -\frac{\lambda\Lambda^2}{32\pi^2},
\end{align}
as second derivative of the potential evaluated at zero is equal to:
\begin{align}
\frac{d^2V}{d\varphi_c^2}\Big|_0 = B+\frac{\lambda\Lambda^2}{32\pi^2}.
\end{align}
Note that, if the derivative would be evaluated at some non zero $\lVEV$, this relation will 
vastly change. \\
After determining $B$, the following criterium for determining $C$ is presented:
\begin{align}
\frac{d^4V}{d\varphi_c^4}\Big|_M=\lambda.
\end{align}
We for now leave (although interesting) discusion of introducing the parameter $M$. \\
The resulting $C$ is:
\begin{align}
C=-\frac{3\lambda^2}{32\pi^2}\left(\ln\frac{\lambda M^2}{2\Lambda^2}+\frac{11}{3}\right).
\end{align}
Note, that resulting $C$ is completly independent of the value of $B$ determined 
earlier\label{CBreason}, as $B$ is not present in $\frac{d^4V}{d\varphi_c^4}\Big|_M$. \\
The resulting potential is:
\begin{align}
V=\frac{\lambda}{4!}\varphi_c^4+\frac{\lambda^2\varphi_c^4}{256\pi^2}\left(\ln
\frac{\varphi_c^2}{M^2}-\frac{25}{6}\right).
\end{align}
Then, the discussion proceeds, that, at $\varphi_c = 0$ we have now maximum, not minimum, 
and that the minimum of the potential occurs at the value of: 
$\varphi_c$ determined by:
\begin{align}
\lambda\ln\frac{\lVEV^2}{M^2}=-\frac{32}{3}\pi^2+O(\lambda),
\end{align}
which is very far outside the expected range of validity of the one-loop approximation and must be 
rejected as superficial. \\
\textit{I am not sure what are the implications of this in physics.}
\section{Scalar electrodynamics}
Now, we proceed, to present treatment of the theorie of our main interest conducted in 
\cite{Coleman1973}. \\
They start with the theory with the lagransian \cite{Coleman1973}(4.1):
\begin{align}\label{CW_electrodynamics}
\mathcal{L} = -\frac{1}{4}(F_{\mu\nu})^2+\frac{1}{2}(\partial_\mu\varphi_1-eA_\mu\varphi_2)^2 + 
\frac{1}{2}(\partial_\mu\varphi_2+eA_\mu\varphi_1)^2-\frac{\lambda}{4!}(\varphi_1^2+\varphi_2^2)^2 
 + \textrm{counterterms}.
\end{align}

Then, the resulting renormalised potential is presented \cite{Coleman1973}(4.5):
\begin{align}
V = \frac{\lambda}{4!}\varphi_c^4+\left(\frac{5\lambda^2}{1152\pi^2}+\frac{3e^4}{64\pi^2}\right)
\varphi_c^4\left(\ln\frac{\varphi_c^2}{M^2}-\frac{25}{6}\right),
\end{align}
whith only a brief note, that it is obtained after "straightforward computation" 
%and it is implied that the procedure was the same as in the $\lambda\varphi^4$ case
.\\
We will now investigate more carefouly this omited step
%and whether it was indeed the same as 
%presented in the previous case, as well as, whether this procedure is eligible for this theory.
.\\
It is implied, that the procedure was the same as in the $\lambda\varphi^4$ case and we shall see, 
whether it was indeed the same, as well as, whether applied procedure is eligible for this theory.\\

%Let us go step-by-step through the process. 
Let us start with the effective potential with not yet calculated integrals and not yet evaluated 
renormalisation constants:
\begin{align}
V = &\frac{\lambda}{4!}\varphi_c^4-\frac{1}{2}B\varphi_c^2-\frac{1}{4!}C\varphi_c^4  \notag \\
&+i\int\frac{d^4k}{(2\pi)^4}\sum\limits_{n=1}^\infty\frac{1}{2n}
\left(\frac{\frac{1}{2}\lambda\varphi_c^2}{k^2+i\epsilon}\right)^n  \notag \\
&+i\int\frac{d^4k}{(2\pi)^4}\sum\limits_{n=1}^\infty\frac{1}{2n}
\left(\frac{\frac{1}{6}\lambda\varphi_c^2}{k^2+i\epsilon}\right)^n  \notag \\
&+i\int\frac{d^4k}{(2\pi)^4}3\sum\limits_{n=1}^\infty\frac{1}{2n}
\left(\frac{e^2\varphi_c^2}{k^2+i\epsilon}\right)^n.
\end{align}
This can be transformed as previously from this infrared divergent form, to the form woth 
singularity only at $\varphi_c=0$:
\begin{align}
V=&\frac{\lambda}{4!}\varphi_c^4-\frac{1}{2}B\varphi_c^2-\frac{1}{4!}C\varphi_c^4 + \notag \\
&\frac{1}{2}\int\frac{d^4k}{(2\pi)^4}\ln\left(1+\frac{\lambda\varphi_c^2}{2k^2}\right) + \notag \\
&\frac{1}{2}\int\frac{d^4k}{(2\pi)^4}\ln\left(1+\frac{\lambda\varphi_c^2}{6k^2}\right) + \notag \\
&\frac{1}{2}\int\frac{d^4k}{(2\pi)^4}3\ln\left(1+\frac{e^2\varphi_c^2}{k^2}\right),
\end{align}
and then calculated using cut-off method at $k^2=\Lambda^2$:
\begin{align}
V=&\frac{\lambda}{4!}\varphi_c^4+\frac{1}{2}B\varphi_c^2+\frac{1}{4!}C\varphi_c^4+ \notag \\
&\frac{\lambda\Lambda^2}{64\pi^2}\varphi_c^2+\frac{\lambda^2\varphi_c^4}{256\pi^2}
\left(\ln\frac{\lambda\varphi_c^2}{2\Lambda^2}-\frac{1}{2}\right) + \notag \\
&\frac{\lambda\Lambda^2}{3\cdot 64\pi^2}\varphi_c^2+\frac{\lambda^2\varphi_c^4}{9\cdot 256\pi^2}
\left(\ln\frac{\lambda\varphi_c^2}{6\Lambda^2}-\frac{1}{2}\right) + \notag \\
&\frac{3e^2\Lambda^2}{32\pi^2}\varphi_c^2+\frac{3e^4\varphi_c^4}{64\pi^2}
\left(\ln\frac{e^2\varphi_c^2}{\Lambda^2}-\frac{1}{2}\right).
\end{align}
%where $\alpha$ is some numerical constant.

\subsection{Renormalisation}\label{CW_renormalisation_scalar_el}
Now, it starts the fun part. \\
The first imposed renormalisation condition in the previous ($\lambda\varphi^4$) case was:
\begin{align}
\frac{d^2V}{d\varphi_c^2}\Big|_0=0,
\end{align}
where, it was stated, is equivalent, to renormalised mass being zero. \\
But it is only true, when $\lVEV = 0$. In theory we are discussing now (\ref{CW_electrodynamics}) 
we can't really use this as (as it is made apparent later in \cite{Coleman1973}, in (4.8)), the 
whole important later argument is laid upon the fact, that $\lVEV \neq 0$. It also can't be, that 
we are now dealing with tree level VEV, which is $0$, and after that, we pass to using on-loop 
level VEV which is non-zero, as the very VEV used in the renormalised condition, should be one 
of the up to one loop level potential -- thus renormalised one. Or can we? Maybe we can use 
zero $\lVEV$ now, and not zero $\lVEV$ later. Nevertheless, we shall investigate, whether this is 
the case, and whether this was used here. \\
Let us start with the second renormalisation condition, namely:
\begin{align}
\frac{d^4V}{d\varphi_c^4}\Big|_M=\lambda,
\end{align}
and reverse engeener what happend in \cite{Coleman1973}. \\
We can do this, because of the reason stated in \ref{CBreason} as $B$ does not appear in 
$\frac{d^4V}{d\varphi_c^4}\Big|_M$. \\
We have that:
\begin{align}
\frac{d^4V}{d\varphi_c^4}\Big|_M = &\lambda + C + \notag \\
&\frac{11\lambda^2}{32\pi^2}+\frac{3\lambda^2}{32\pi^2}\ln\frac{\lambda M^2}{2\Lambda^2}+\notag \\
&\frac{11\lambda^2}{288\pi^2}+\frac{\lambda^2}{96\pi^2}\ln\frac{\lambda M^2}{6\Lambda^2}+\notag \\
&\frac{(75-18\alpha)e^4}{16\pi^2}+\frac{9e^4}{8\pi^2}\ln\frac{e^2M^2}{\Lambda^2}. 
\end{align}
From this, for $\frac{d^4V}{d\varphi_c^4}\Big|_M=\lambda$, we conclude that:
\begin{align}
C = -\Big(
&\frac{11\lambda^2}{32\pi^2}+\frac{3\lambda^2}{32\pi^2}\ln\frac{\lambda M^2}{2\Lambda^2}+\notag \\
&\frac{11\lambda^2}{288\pi^2}+\frac{\lambda^2}{96\pi^2}\ln\frac{\lambda M^2}{6\Lambda^2}+\notag \\
&\frac{(75-18\alpha)e^4}{16\pi^2}+\frac{9e^4}{8\pi^2}\ln\frac{e^2M^2}{\Lambda^2}
\Big).
\end{align}
Subsituing this result to the potential result in:
\begin{align}
V=&\frac{\lambda}{4!}\varphi_c^4+\frac{1}{2}B\varphi_c^2- \notag \\
&\frac{1}{4!}\Big(
\frac{11\lambda^2}{32\pi^2}+\frac{3\lambda^2}{32\pi^2}\ln\frac{\lambda M^2}{2\Lambda^2}+\notag \\
&\frac{11\lambda^2}{288\pi^2}+\frac{\lambda^2}{96\pi^2}\ln\frac{\lambda M^2}{6\Lambda^2}+\notag \\
&\frac{(75-18\alpha)e^4}{16\pi^2}+\frac{9e^4}{8\pi^2}\ln\frac{e^2M^2}{\Lambda^2}
\Big)\varphi_c^4+ \notag \\
&\frac{\lambda\Lambda^2}{64\pi^2}\varphi_c^2+\frac{\lambda^2\varphi_c^4}{256\pi^2}
\left(\ln\frac{\lambda\varphi_c^2}{2\Lambda^2}-\frac{1}{2}\right) + \notag \\
&\frac{\lambda\Lambda^2}{3\cdot 64\pi^2}\varphi_c^2+\frac{\lambda^2\varphi_c^4}{9\cdot 256\pi^2}
\left(\ln\frac{\lambda\varphi_c^2}{6\Lambda^2}-\frac{1}{2}\right) + \notag \\
&\frac{3e^2\Lambda^2}{32\pi^2}\varphi_c^2+\frac{3e^4\varphi_c^4}{64\pi^2}
\left(\ln\frac{e^2\varphi_c^2}{\Lambda^2}-\alpha\right),
\end{align}
and after canceling:
\begin{align}
V = &\frac{\lambda}{4!}\varphi_c^4+\frac{1}{2}B\varphi_c^2 + \notag \\
&\frac{\lambda\Lambda^2}{64\pi^2}\varphi_c^2+ 
\frac{\lambda\Lambda^2}{3\cdot 64\pi^2}\varphi_c^2+
\frac{3e^2\Lambda^2}{32\pi^2}\varphi_c^2 + \notag \\
&\left(\frac{5\lambda^2}{1152\pi^2}+\frac{3e^4}{64\pi^2}\right)\varphi_c^4
\left(\ln\frac{\varphi_c^2}{M^2}-\frac{25}{6}\right).
\end{align}
%Let us notice, that the constant $\alpha$ disappeared entearly from the expression as it was 
%canceld out. \\
Let us notice, that above potential differs from \cite{Coleman1973}(4.5) only by the terms:
\begin{align}
\frac{1}{2}B\varphi_c^2 + 
\frac{\lambda\Lambda^2}{64\pi^2}\varphi_c^2+ 
\frac{\lambda\Lambda^2}{3\cdot 64\pi^2}\varphi_c^2+
\frac{3e^2\Lambda^2}{32\pi^2}\varphi_c^2. 
\end{align}
As such, in the renormalistion scheme used in \cite{Coleman1973}, we must have that:
\begin{align}
\frac{1}{2}B\varphi_c^2 + 
\frac{\lambda\Lambda^2}{64\pi^2}\varphi_c^2+ 
\frac{\lambda\Lambda^2}{3\cdot 64\pi^2}\varphi_c^2+
\frac{3e^2\Lambda^2}{32\pi^2}\varphi_c^2 = 0, 
\end{align}
thus:
\begin{align}\label{CW_B_dm}
B = -2\Big(
\frac{\lambda\Lambda^2}{64\pi^2}+ 
\frac{\lambda\Lambda^2}{3\cdot 64\pi^2}+
\frac{3e^2\Lambda^2}{32\pi^2}\Big). 
\end{align}
Then, the renormalised potential is indeed:
\begin{align}
V = \frac{\lambda}{4!}\varphi_c^4+
\left(\frac{5\lambda^2}{1152\pi^2}+\frac{3e^4}{64\pi^2}\right)\varphi_c^4
\left(\ln\frac{\varphi_c^2}{M^2}-\frac{25}{6}\right)
\end{align}
as in \cite{Coleman1973}(4.5), but what is important is, then indeed
\begin{align}
\frac{d^2V}{d\varphi_c^2}\Big|_0=0,
\end{align}
so this was preceisly the condition used to renormalised this potential. \\
Now this is the mystery: \\
This condition, in the light of this theory, should not mean the disapperance of the renormalised 
mass, as $\lVEV \neq 0$ in this theory, and the mass is $\frac{d^2V}{d\varphi_c^2}\Big|_\lVEV$. 
As such, what physical meaning this condition gives? 
From the computational point of view, it certainly results in the potential being of order $4$ 
in the field, but physical one? \\
In \ref{onshellphi4} we will investigate the on-shell approuch of this method.













