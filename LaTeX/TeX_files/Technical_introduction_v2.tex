% mainfile: ../master_thesis_GW.tex
\chapter{Technical introduction}

In this chapter we present some general ideas, which summary s included here for e completeness. 
We also set conventions, so the readder may use this chapter as a reference.

\section{Scalar QED}
This is the model of our main interest in this thesis, as it is great example of a simple 
model were radiative symmetry breaking occurs. \\

It is not the case for, for example. ??? \smalltodo \\

This model will be used throughout the whole thesis, unless stated otherwise. \\

For a toy model we choose theory of scalar electrodynamics, described by the Lagrangian:
\begin{equation}
\mathcal{L} = -\frac{1}{4}F_{\mu\nu}F^{\mu\nu}+D_\mu\Phi D^\mu\Phi^\dag-\lambda\Phi^4,
\end{equation}
where $\Phi$ is a complex scalar field and the vector field present is $U(1)$ gauge boson. \\

Writing operator $D$ more explicitly it reads:
\begin{equation}
\mathcal{L} = -\frac{1}{4}F_{\mu\nu}F^{\mu\nu}+(\partial_\mu\Phi + ieA_\mu\Phi) 
(\partial^\mu\Phi^\dag-ieA^\mu\Phi^\dag)-\lambda\Phi^4,
\end{equation}

For the reasons that will be clear in ref{finite momentum} we will write $\Phi$ 
field as two real scalar fields 
$\varphi_1$ and $\varphi_2$, such that:
\begin{equation}
\Phi = \frac{1}{\sqrt{2}}(\varphi_1+\varphi_2)
\end{equation}
Then Lagrangian takes form:
\begin{align}
\mathcal{L} = -\frac{1}{4}F_{\mu\nu}F^{\mu\nu}+ \frac{1}{2}(\partial_\mu\varphi_1 - eA_\mu\varphi_2)
(\partial^\mu\varphi_1 - eA^\mu\varphi_2)  \\
+\frac{1}{2}(\partial_\mu\varphi_2+eA_\mu\varphi_1)(\partial^\mu\varphi_2+eA^\mu\varphi_1) 
-\frac{1}{4}\lambda(\varphi_1^2+\varphi_2^2)^2,\notag
\end{align}

which we will write for brevity as:
\begin{align}
\mathcal{L} = -\frac{1}{4}F_{\mu\nu}F^{\mu\nu}+ 
\frac{1}{2}(\partial_\mu\varphi_1 - eA_\mu\varphi_2)^2\\
+\frac{1}{2}(\partial_\mu\varphi_2+eA_\mu\varphi_1)^2
-\frac{1}{4}\lambda(\varphi_1^2+\varphi_2^2)^2. \notag
\end{align}


For a better track of what is independent of numerical convention, we will also write:
\begin{align}\label{scalar_electrodynamics}
\mathcal{L} = -\frac{1}{4}F_{\mu\nu}F^{\mu\nu}+ 
\frac{1}{2}(\partial_\mu\varphi_1 - eA_\mu\varphi_2)^2\\
+\frac{1}{2}(\partial_\mu\varphi_2+eA_\mu\varphi_1)^2
-\cl\lambda(\varphi_1^2+\varphi_2^2)^2, \notag
\end{align}
but $\cl = \frac{1}{4}$ everywhere in the thesis if not stated otherwise.
\section{Effective potential}
\todo{}
Some words about what it is and why we are interested in it.


\subsection{Calculation of the unrenormalized effective potential}

Tree level potential in our model is:
\begin{equation}
V_T = \cl\lambda(\varphi_1^2+\varphi_2^2)^2.
\end{equation}


The one loop correction to the effective potential is calculated as a sum of the following diagrams:
\\
\begin{tikzpicture}[line width=1.5 pt, scale=1.3]
    \begin{scope}
%	    \draw[scalar] (0:1)--(0,0);
        \draw[scalar] (0,-0.5) --(1.5,-0.5);
	    \draw[scalar] (1,0) arc (180:-180:.5);
	    %\draw[vector] (2,0) arc (0:-180:.5);
	    \draw[scalar] (1.5,-0.5) --(3,-0.5);
	    \node at (-.2,-0.5) {$\varphi_1$};
	    \node at (3.2,-0.5) {$\varphi_1$};
%	    \node at (40:2) {$\varphi_1$};
        \node at (1.5,0.8) {$\varphi_1$};
    \end{scope},
%\end{tikzpicture}
%\begin{tikzpicture}[line width=1.5 pt, scale=1.3]
    \node at (3.6, -0.5) {$\textrm{, }$};
    \begin{scope}[shift = {(3.5,0)}]
    %	\draw[scalar] (0:1)--(0,0);
        \draw[scalar] (1,1) -- (1.5,0.5); 
    %    \draw[scalar] (1.5,0.5) -- (1,1); 
        \draw[scalar] (1.5,0.5) -- (2,1); 
	    \draw[scalar] (1,0) arc (180:-180:.5);
    %	\draw[vector] (2,0) arc (0:-180:.5);
        \draw[scalar] (1,-1)--(1.5,-0.5);
	    \draw[scalar] (1.5,-0.5)--(2,-1);
	    \node at (0.8,-1.2) {$\varphi_1$};
	    \node at (2.2,-1.2) {$\varphi_1$};
	    \node at (0.8,1.2) {$\varphi_1$};
	    \node at (2.2,1.2) {$\varphi_1$};
	    \node at (0.75,0) {$\varphi_1$};
    %	\node at (40:2) {$\varphi_1$};
        
    \end{scope}
    \node at (6.2, -0.5) {$\textrm{, }$};
    \begin{scope}[shift = {(6.5,0)}]
    %	\draw[scalar] (0:1)--(0,0);
        \draw[scalar] (1,1) -- (1.5,0.5); 
    %    \draw[scalar] (1.5,0.5) -- (1,1); 
        \draw[scalar] (1.5,0.5) -- (2,1); 
	    \draw[scalar] (1,0) arc (180:-180:.5);
    %	\draw[vector] (2,0) arc (0:-180:.5);
%        \draw[scalar] (1,-1)--(1.5,-0.5);
	    \draw[scalar] (1.9,-0.3)--(2.7,-0.5);
	    \draw[scalar] (2.2, -1.1)--(1.9,-0.3);
	    \node at (2.4,-1.3) {$\varphi_1$};
	    \node at (2.9,-0.6) {$\varphi_1$};
	    \node at (0.8,1.2) {$\varphi_1$};
	    \node at (2.2,1.2) {$\varphi_1$};
	    \node at (0.75,0.3) {$\varphi_1$};
    %	\node at (40:2) {$\varphi_1$};
        \draw[scalar] (1.1,-0.3)--(0.3,-0.5);
	    \draw[scalar] (0.8, -1.1)--(1.1,-0.3);
	    \node at (0.6,-1.3) {$\varphi_1$};
	    \node at (0.1,-0.6) {$\varphi_1$};
        
    \end{scope}
    \node at (10, -0.5) {$\textrm{, \ .\ .\ .\ ,}$};
    
\end{tikzpicture} \\

\begin{tikzpicture}[line width=1.5 pt, scale=1.3]
    \begin{scope}
%	    \draw[scalar] (0:1)--(0,0);
        \draw[scalar] (0,-0.5) --(1.5,-0.5);
	    \draw[scalar] (1,0) arc (180:-180:.5);
	    %\draw[vector] (2,0) arc (0:-180:.5);
	    \draw[scalar] (1.5,-0.5) --(3,-0.5);
	    \node at (-.2,-0.5) {$\varphi_1$};
	    \node at (3.2,-0.5) {$\varphi_1$};
%	    \node at (40:2) {$\varphi_1$};
        \node at (1.5,0.8) {$\varphi_2$};
    \end{scope},
%\end{tikzpicture}
%\begin{tikzpicture}[line width=1.5 pt, scale=1.3]
    \node at (3.6, -0.5) {$\textrm{, }$};
    \begin{scope}[shift = {(3.5,0)}]
    %	\draw[scalar] (0:1)--(0,0);
        \draw[scalar] (1,1) -- (1.5,0.5); 
    %    \draw[scalar] (1.5,0.5) -- (1,1); 
        \draw[scalar] (1.5,0.5) -- (2,1); 
	    \draw[scalar] (1,0) arc (180:-180:.5);
    %	\draw[vector] (2,0) arc (0:-180:.5);
        \draw[scalar] (1,-1)--(1.5,-0.5);
	    \draw[scalar] (1.5,-0.5)--(2,-1);
	    \node at (0.8,-1.2) {$\varphi_1$};
	    \node at (2.2,-1.2) {$\varphi_1$};
	    \node at (0.8,1.2) {$\varphi_1$};
	    \node at (2.2,1.2) {$\varphi_1$};
	    \node at (0.75,0) {$\varphi_2$};
    %	\node at (40:2) {$\varphi_1$};
        
    \end{scope}
    \node at (6.2, -0.5) {$\textrm{, }$};
    \begin{scope}[shift = {(6.5,0)}]
    %	\draw[scalar] (0:1)--(0,0);
        \draw[scalar] (1,1) -- (1.5,0.5); 
    %    \draw[scalar] (1.5,0.5) -- (1,1); 
        \draw[scalar] (1.5,0.5) -- (2,1); 
	    \draw[scalar] (1,0) arc (180:-180:.5);
    %	\draw[vector] (2,0) arc (0:-180:.5);
%        \draw[scalar] (1,-1)--(1.5,-0.5);
	    \draw[scalar] (1.9,-0.3)--(2.7,-0.5);
	    \draw[scalar] (2.2, -1.1)--(1.9,-0.3);
	    \node at (2.4,-1.3) {$\varphi_1$};
	    \node at (2.9,-0.6) {$\varphi_1$};
	    \node at (0.8,1.2) {$\varphi_1$};
	    \node at (2.2,1.2) {$\varphi_1$};
	    \node at (0.75,0.3) {$\varphi_2$};
    %	\node at (40:2) {$\varphi_1$};
        \draw[scalar] (1.1,-0.3)--(0.3,-0.5);
	    \draw[scalar] (0.8, -1.1)--(1.1,-0.3);
	    \node at (0.6,-1.3) {$\varphi_1$};
	    \node at (0.1,-0.6) {$\varphi_1$};
        
    \end{scope}
    \node at (10, -0.5) {$\textrm{, \ .\ .\ .\ ,}$};
    
\end{tikzpicture} \\

\begin{tikzpicture}[line width=1.5 pt, scale=1.3]
    \begin{scope}
%	    \draw[scalar] (0:1)--(0,0);
        \draw[scalar] (0,-0.5) --(1.5,-0.5);
	    \draw[vector] (1,0) arc (180:-180:.5);
	    %\draw[vector] (2,0) arc (0:-180:.5);
	    \draw[scalar] (1.5,-0.5) --(3,-0.5);
	    \node at (-.2,-0.5) {$\varphi_1$};
	    \node at (3.2,-0.5) {$\varphi_1$};
%	    \node at (40:2) {$\varphi_1$};
        \node at (1.5,0.8) {$\gamma$};
    \end{scope},
%\end{tikzpicture}
%\begin{tikzpicture}[line width=1.5 pt, scale=1.3]
    \node at (3.6, -0.5) {$\textrm{, }$};
    \begin{scope}[shift = {(3.5,0)}]
    %	\draw[scalar] (0:1)--(0,0);
        \draw[scalar] (1,1) -- (1.5,0.5); 
    %    \draw[scalar] (1.5,0.5) -- (1,1); 
        \draw[scalar] (1.5,0.5) -- (2,1); 
	    \draw[vector] (1,0) arc (180:-180:.5);
    %	\draw[vector] (2,0) arc (0:-180:.5);
        \draw[scalar] (1,-1)--(1.5,-0.5);
	    \draw[scalar] (1.5,-0.5)--(2,-1);
	    \node at (0.8,-1.2) {$\varphi_1$};
	    \node at (2.2,-1.2) {$\varphi_1$};
	    \node at (0.8,1.2) {$\varphi_1$};
	    \node at (2.2,1.2) {$\varphi_1$};
	    \node at (0.75,0) {$\gamma$};
    %	\node at (40:2) {$\varphi_1$};
        
    \end{scope}
    \node at (6.2, -0.5) {$\textrm{, }$};
    \begin{scope}[shift = {(6.5,0)}]
    %	\draw[scalar] (0:1)--(0,0);
        \draw[scalar] (1,1) -- (1.5,0.5); 
    %    \draw[scalar] (1.5,0.5) -- (1,1); 
        \draw[scalar] (1.5,0.5) -- (2,1); 
	    \draw[vector] (1,0) arc (180:-180:.5);
    %	\draw[vector] (2,0) arc (0:-180:.5);
%        \draw[scalar] (1,-1)--(1.5,-0.5);
	    \draw[scalar] (1.9,-0.3)--(2.7,-0.5);
	    \draw[scalar] (2.2, -1.1)--(1.9,-0.3);
	    \node at (2.4,-1.3) {$\varphi_1$};
	    \node at (2.9,-0.6) {$\varphi_1$};
	    \node at (0.8,1.2) {$\varphi_1$};
	    \node at (2.2,1.2) {$\varphi_1$};
	    \node at (0.75,0.3) {$\gamma$};
    %	\node at (40:2) {$\varphi_1$};
        \draw[scalar] (1.1,-0.3)--(0.3,-0.5);
	    \draw[scalar] (0.8, -1.1)--(1.1,-0.3);
	    \node at (0.6,-1.3) {$\varphi_1$};
	    \node at (0.1,-0.6) {$\varphi_1$};
        
    \end{scope}
    \node at (10, -0.5) {$\textrm{, \ .\ .\ ..}$};
    
\end{tikzpicture}

where: \\
%\hspace*{2cm}
\begin{tikzpicture}[line width=1.5 pt, scale=1.3]
\begin{scope}[shift = {(6.5,0)}]
    %	\draw[scalar] (0:1)--(0,0);
        \draw[scalar] (1,1) -- (1.5,0.5); 
    %    \draw[scalar] (1.5,0.5) -- (1,1); 
        \draw[scalar] (1.5,0.5) -- (2,1); 
	    \draw[scalar] (1,0) arc (180:-180:.5);
    %	\draw[vector] (2,0) arc (0:-180:.5);
%        \draw[scalar] (1,-1)--(1.5,-0.5);
	    \draw[scalar] (1.9,-0.3)--(2.7,-0.5);
	    \draw[scalar] (2.2, -1.1)--(1.9,-0.3);
	    \node at (2.4,-1.3) {$\varphi_1$};
	    \node at (2.9,-0.6) {$\varphi_1$};
	    \node at (0.8,1.2) {$\varphi_1$};
	    \node at (2.2,1.2) {$\varphi_1$};
	    \node at (0.75,0.3) {$\varphi_1$};
    %	\node at (40:2) {$\varphi_1$};
        \draw[scalar] (1.1,-0.3)--(0.3,-0.5);
	    \draw[scalar] (0.8, -1.1)--(1.1,-0.3);
	    \node at (0.6,-1.3) {$\varphi_1$};
	    \node at (0.1,-0.6) {$\varphi_1$};
        \node at (2.6,0.5) {$n$ pairs};
        \node at (2.9,0.1) {of $\varphi_1$ legs};
    \end{scope}
\end{tikzpicture}

\vspace*{-3cm}\begin{align}
\hspace*{3cm}=i\int \frac{\textrm{d}^4k}{(2\pi)^4}\frac{1}{2n}\left(\frac{12\cl\lambda
\varphi_1^2}
{k^2+i\varepsilon}\right)^n 
\end{align} \\[8pt]

\begin{tikzpicture}[line width=1.5 pt, scale=1.3]
\begin{scope}[shift = {(6.5,0)}]
    %	\draw[scalar] (0:1)--(0,0);
        \draw[scalar] (1,1) -- (1.5,0.5); 
    %    \draw[scalar] (1.5,0.5) -- (1,1); 
        \draw[scalar] (1.5,0.5) -- (2,1); 
	    \draw[scalar] (1,0) arc (180:-180:.5);
    %	\draw[vector] (2,0) arc (0:-180:.5);
%        \draw[scalar] (1,-1)--(1.5,-0.5);
	    \draw[scalar] (1.9,-0.3)--(2.7,-0.5);
	    \draw[scalar] (2.2, -1.1)--(1.9,-0.3);
	    \node at (2.4,-1.3) {$\varphi_1$};
	    \node at (2.9,-0.6) {$\varphi_1$};
	    \node at (0.8,1.2) {$\varphi_1$};
	    \node at (2.2,1.2) {$\varphi_1$};
	    \node at (0.75,0.3) {$\varphi_2$};
    %	\node at (40:2) {$\varphi_1$};
        \draw[scalar] (1.1,-0.3)--(0.3,-0.5);
	    \draw[scalar] (0.8, -1.1)--(1.1,-0.3);
	    \node at (0.6,-1.3) {$\varphi_1$};
	    \node at (0.1,-0.6) {$\varphi_1$};
        \node at (2.6,0.5) {$n$ pairs};
        \node at (2.9,0.1) {of $\varphi_1$ legs};
    \end{scope}
\end{tikzpicture}

\vspace*{-3cm}\begin{align}
\hspace*{3cm}=i\int \frac{\textrm{d}^4k}{(2\pi)^4}\frac{1}{2n}&\left(
\frac{1}{3}\frac{12\cl\lambda\varphi_1^2}
{k^2+i\varepsilon}\right)^n 
\end{align} \\[8pt]

\begin{tikzpicture}[line width=1.5 pt, scale=1.3]
\begin{scope}[shift = {(6.5,0)}]
    %	\draw[scalar] (0:1)--(0,0);
        \draw[scalar] (1,1) -- (1.5,0.5); 
    %    \draw[scalar] (1.5,0.5) -- (1,1); 
        \draw[scalar] (1.5,0.5) -- (2,1); 
	    \draw[vector] (1,0) arc (180:-180:.5);
    %	\draw[vector] (2,0) arc (0:-180:.5);
%        \draw[scalar] (1,-1)--(1.5,-0.5);
	    \draw[scalar] (1.9,-0.3)--(2.7,-0.5);
	    \draw[scalar] (2.2, -1.1)--(1.9,-0.3);
	    \node at (2.4,-1.3) {$\varphi_1$};
	    \node at (2.9,-0.6) {$\varphi_1$};
	    \node at (0.8,1.2) {$\varphi_1$};
	    \node at (2.2,1.2) {$\varphi_1$};
	    \node at (0.75,0.3) {$\gamma$};
    %	\node at (40:2) {$\varphi_1$};
        \draw[scalar] (1.1,-0.3)--(0.3,-0.5);
	    \draw[scalar] (0.8, -1.1)--(1.1,-0.3);
	    \node at (0.6,-1.3) {$\varphi_1$};
	    \node at (0.1,-0.6) {$\varphi_1$};
        \node at (2.6,0.5) {$n$ pairs};
        \node at (2.9,0.1) {of $\varphi_1$ legs};
    \end{scope}
\end{tikzpicture}

\vspace*{-3cm}\begin{align}
\hspace*{5cm}=i\int \frac{\textrm{d}^4k}{(2\pi)^4}\frac{1}{2n}&\left(\frac{2e^2\frac{1}{2}\varphi_1^2}
{k^2+i\varepsilon}\right)^n(g^\mu_{\ \,\mu} - 1).
\end{align} \\[8pt]

Summing all the diagrams in series gives:


\begin{align}
V_{1L} = i\int \frac{\textrm{d}^4k}{(2\pi)^4}\sum\limits_{n=1}^{\infty}
\frac{1}{2n}&\left(\frac{12\cl\lambda\varphi_1^2}
{k^2+i\varepsilon}\right)^n + \\
i\int \frac{\textrm{d}^4k}{(2\pi)^4}\sum\limits_{n=1}^{\infty}\frac{1}{2n}&
\left(\frac{1}{3}\frac{12\cl\lambda\varphi_1^2}
{k^2+i\varepsilon}\right)^n +\\
i\int \frac{\textrm{d}^4k}{(2\pi)^4}\sum\limits_{n=1}^{\infty}\frac{1}{2n}&
\left(\frac{2e^2\frac{1}{2}\varphi_1^2}
{k^2+i\varepsilon}\right)^n(g^\mu_{\ \,\mu} - 1).
\end{align}\label{divergent sums}

These integrals seems to be hideously infrared divergent. 
We can, however Wick rotate them to the Euklidean space and perform summation to get:
\begin{align}\label{eff_before_reg}
V_{1L}=&\frac{1}{2} \int\frac{d^4k}{(2\pi)^4}\ln{\left(1+\frac{12\cl\lambda\varphi_1^2}{k^2}
\right)} +\\
&\frac{1}{2} \int\frac{d^4k}{(2\pi)^4}\ln{\left(1+\frac{4\cl\lambda\varphi_1^2}{k^2}
\right)} +\\ 
&\frac{3}{2} \int\frac{d^4k}{(2\pi)^4}\ln{\left(1+\frac{e^2\varphi_1^2}{k^2}\right)}
\end{align}
Now, integrals have only singularity at $\varphi_1 = 0$. \\

There are now several ways to perform these integrals. We will present here briefly two of them: 
cut-off and
dimentional regularisation (DR). \\

For all of our further calculations in the thesis, we will use DR, 
but cut-off will apear in \ref{CWchapter} 
as a method used in \cite{Coleman1973}. \\
\subsubsection{Cut-off}
The idea of this method is to restrict the integration to $k^2 \leq \Lambda$, for a 
parameter $\Lambda$. \\

Integrating \ref{eff_before_reg} up to $\Lambda$ gives:
\begin{align}
V_{1L}=
&\frac{3\cl\lambda\Lambda^2}{8\pi^2}\varphi_c^2+\frac{9\cl\lambda^2\varphi_c^4}{4\pi^2}
\left(\ln\frac{12\cl\lambda\varphi_c^2}{\Lambda^2}-\frac{1}{2}\right)  \notag \\
+&\frac{\cl\lambda\Lambda^2}{8\pi^2}\varphi_c^2+\frac{\cl\lambda^2\varphi_c^4}{4\pi^2}
\left(\ln\frac{4\cl\lambda\varphi_c^2}{\Lambda^2}-\frac{1}{2}\right)  \notag \\
+&\frac{3e^2\Lambda^2}{32\pi^2}\varphi_c^2+\frac{3e^4\varphi_c^4}{64\pi^2}
\left(\ln\frac{e^2\varphi_c^2}{\Lambda^2}-\frac{1}{2}\right).
\end{align}

\subsubsection{Dimentional regularisation}
The idea of this method is to observe, that the divergence is only in four dimentions 
and if we formally change expresions to $4-2\epsilon$ dimentions. 
It is performed by treating the integral as a linear functional depending on two parameters --
function and dimention given by some specific formula. Then we can calculate it with our 
function of interest and $D = 4-2\epsilon$ as arguments.  \\

%$D = 4-2\epsilon$
After passing to $D=4-2\epsilon$ dimensions the integrals take form: 
\begin{align}
V_{1L}=&\frac{\mu^\epsilon}{2} \int\frac{d^Dk}{(2\pi)^D}\ln{\left(1+\frac{12\cl\lambda\varphi_1^2}{k^2}
\right)}\\
+&\frac{\mu^\epsilon}{2} \int\frac{d^Dk}{(2\pi)^D}\ln{\left(1+\frac{4\cl\lambda\varphi_1^2}{k^2}
\right)} \\ 
+&\frac{(D-1)\mu^\epsilon}{2} \int\frac{d^Dk}{(2\pi)^D}\ln{\left(1+\frac{e^2\varphi_1^2}{k^2}\right)}.
\end{align}
This introduces the parameter $\mu$, \smalltodo{more about $\mu$}. \\

After calculating the integrals in $D$ dimentions we have:
\begin{align}
V_{1L}=&\frac{1}{4}\frac{(12\cl\lambda\varphi_1^2)^2}{(4\pi)^2}\left(-\frac{2}{\epsilon}
+\gamma_E-\frac{3}{2}+\log{\frac{(12\cl\lambda\varphi_1^2)^2}{4\pi\mu^2}}\right)  \\
+&\frac{1}{4}\frac{(4\cl\lambda\varphi_1^2)^2}{(4\pi)^2}\left(-\frac{2}{\epsilon}
+\gamma_E-\frac{3}{2}+\log{\frac{(4\cl\lambda\varphi_1^2)^2}{4\pi\mu^2}}\right)  \\
+&\frac{1}{4}\frac{3(e^2\varphi_1^2)^2}{(4\pi)^2}\left(-\frac{2}{\epsilon}+\gamma_E
-\frac{5}{6}+\log{\frac{e^2\varphi^2}{4\pi\mu^2}}\right).
\end{align}

\todo{dopisac potencjał z kontrczłonami}

\section{Renormalisation of the effective potential}
The task that is still left, is to mange the infinities that come when we pass to the limit 
and:
\begin{itemize} 
\item take $\Lambda \to \infty$, allowing for arbitrary large $k^2$ in the case of the cut-off
\item take $\epsilon \to 0$ and returning to $4$ dimentions in the case of DR.

\end{itemize}
This is the scope of renormalisation and is the main interest of this thesis. 
Different renormalisation techniques should ... \todo{how this is exactly}

It is the goal of this thesis to show wether two of them -- \MSbar and On shell give same results.


\subsection{\texorpdfstring{\MSbar}{MS-bar} renormalisation of the effective potential}
Although the goal of our thesis is to renormalise effective potential on-shell

\MSbar\ renormalisation gives:
\begin{align}
V_{R}=&\cl\lambda\varphi_1^4 + \notag \\
+&\frac{1}{4}\frac{(12\cl\lambda\varphi_1^2)^2}{(4\pi)^2}\left(-\frac{3}{2}+ 
\log{\frac{(12\cl\lambda\varphi_1^2)^2}{\mu^2}}\right)\notag  \\
+&\frac{1}{4}\frac{(4\cl\lambda\varphi_1^2)^2}{(4\pi)^2}\left(-\frac{3}{2}+ 
\log{\frac{(4\cl\lambda\varphi_1^2)^2}{\mu^2}}\right) \notag \\
+&\frac{1}{4}\frac{3(e^2\varphi_1^2)^2}{(4\pi)^2}\left(-\frac{5}{6}+
\log{\frac{e^2\varphi_1^2}{\mu^2}}\right).
\end{align}

Here, as will be apparent in the second, we are interested only in $e^4$ part, so from now on, 
we will write:
\begin{align}
V_{R}=\cl\lambda\varphi_1^4 + \frac{3e^4}{64\pi^2}&\left(-\frac{5}{6}+
\log{\frac{e^2\varphi_1^2}{\mu^2}}\right)\varphi_1^4.
\end{align}

We can bind $e$ to $\lambda$ and $v$ at the loop level from the VEV definition:
\begin{align}
\DphiI\VR \Big|_v= 0.
\end{align}
This gives the condition:
\begin{align}
4\cl\lambda v^3-\frac{e^4v^3}{16\pi^2}-\frac{3e^4v^3}{16\pi^2}\log\frac{e^2v^2}{\mu^2}=0.
\end{align}
Setting scale parameter $\mu$ to the effective mass of the vector, namely $ev$, we have 
simpler form of:
\begin{align}
4\cl\lambda v^3-\frac{e^4v^3}{16\pi^2}=0.
\end{align}
Which gives:
\begin{align}\label{e4_lambda_MS}
\lambda = \frac{e^4}{64\cl\pi^2}.
\end{align}
Writing potential with this substitutions yields:

\begin{align}\label{MSbar_result}
V_R = \frac{3e^4}{64\pi^2}\Big(-\frac{1}{2}+\log\frac{\varphi_1^2}{v^2}\Big)\varphi_1^4.
\end{align}
This is exactly the same potential as obtained in \cite{Coleman1973} with different renormalisation.

Equations \ref{e4_lambda_MS} is very important in this discussion as it states, that $\lambda$ 
is of order $e^4$ in our model. This post factum justifies our 
choice in taking only $e^4$ part, as other part was of order $e^8$. \\

\subsubsection{Running mass}\label{MSbar running mass}
\todo{Zrobić tu audyt merytoryczny}
We can now also bind square of the running mass $M_R^2$ as the second derivative of the 
renormalised effective potential at VEV.

\begin{align}
M_R^2 = \DphiII\VR \Big|_v = \frac{9e^4}{16\pi^2}\Big(-\frac{1}{2}\Big)v^2+\frac{3e^4v^2}{8\pi^2} 
+\frac{9e^4v^2}{32\pi^2}.
\end{align}
This gives that:
\begin{align}
M_R^2 = \frac{3e^4v^2}{8\pi^2}.
\end{align}

From this we have that the ratio between scalar mass $M_P^2$ and vector mass $m(V)^2$ is:
\begin{align}
\frac{M_R^2}{m(V)^2} = \frac{\frac{3e^4v^2}{8\pi^2}}{e^2v^2} = \frac{3e^2}{8\pi^2}.
\end{align}

We can now also express $e^4$ and $\lambda$ in terms of $M_P^2$ and $v$:
\begin{align}
e^4 = \frac{8M_R^2\pi^2}{3v^2} \\
\lambda = \frac{M_R^2}{24\cl v^2}.
\end{align}

Writing potential in these terms gives:
\begin{align}
V_R = \frac{M_R^2}{8v^2}\Big(-\frac{1}{2}+\log\frac{\varphi_1^2}{v^2}\Big)\varphi_1^4.
\end{align}

\todo{COś o stałym renormalizacji tu napisać}

\subsection{On-shell renormalisation of the effective potential}
\todo{dac tutaj motywacjąę do on-shell i napisać że to jest główne zagadnienie i że to 
będzie właśnie zaraz omówione}
napisać tez o on shell jako takim -- czym jest, jakie są warunki itd.














