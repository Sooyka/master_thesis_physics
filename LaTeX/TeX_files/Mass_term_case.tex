% mainfile: ../master_thesis_GW.tex
\chapter{Introduction -- mass term case}
For comparison, we present here results for above methods used in the case with explicit mass term.

\section{On-shell finite momentum approach}
\todo{przeredagować to, bo na razie bez sensu, że jeest pierwsze}
For the comparison, we will present the same calculation, performed on the analoguos theory with 
explicit mass term. Similary as in the \ref{no_mass_term} we need to shift fields for 
\ref{second_derivativ_condition} to be satisfied. The Lagrangian in this case is:
\begin{align}\label{mass_term_lagrangian}
\mathcal{L} = &-\frac{1}{4}F_{\mu\nu}F^{\mu\nu}+ 
\frac{1}{2}(\partial_\mu(\varphi_1+v) - eA_\mu\varphi_2)^2\\
&+\frac{1}{2}(\partial_\mu\varphi_2+eA_\mu(\varphi_1+v))^2 - \cm m^2((\varphi_1+v)^2+\varphi_2^2)
-\cl\lambda((\varphi_1+v)^2+\varphi_2^2)^2. \notag
\end{align}
With renormalisation constatnts:
\begin{align}
\mathcal{L_R} = &-\frac{1}{4}F_{\mu\nu}F^{\mu\nu} \\
&+(1+ \dZ)(\frac{1}{2}(\partial_\mu(\varphi_1+v) - eA_\mu\varphi_2)^2
+\frac{1}{2}(\partial_\mu\varphi_2+eA_\mu(\varphi_1+v))^2) \notag \\
&-(1+\dZ)^2\cl(\lambda+\dl)((\varphi_1+v)^2+\varphi_2^2)^2  \notag \\
&-(1+\dZ)c_m(m+\dm)((\varphi_1+v)^2+\varphi_2^2).
\end{align}
Corrections then are:
\begin{align}
\dS&=-12\cl v^2(2\lambda\dZphi+\dl)-2\cm\dm-2\cm m^2\dZ -p^2\dZ \\
\dT&=-4\cl v^3(2\lambda\dZphi+\dl)-2\cm\dm v-2\cm m^2v\dZ.
\end{align}
This changes the form of renormalisation constants to:
\begin{align}
\dm &= \frac{-1}{4\cm}\left(\Rep{\Sigma}-\frac{3}{v}T-(4\cm m^2+M_P^2)\Rep{\Sigma'}
\right)\label{delta_m_mass}\\
\dl &= \frac{1}{8\cl v^2}\left(\Rep{\Sigma}-\frac{1}{v}T-(16\cl\lambda v^2+M_P^2)
\Rep{\Sigma'}
\right)\\
\dZ &= \Rep{\Sigma'}.
\end{align}
The only difference is $4\cm m^2$ term in \ref{delta_m_mass}. 
\section{"Zero momentum" approach}
Here we will compare two kinds of "zero momentum" approach. 
First will be imposing renormalisation conditions in terms of only derivatives of 
effective potential. This is the zero momentum approach as first and second derivatives 
are limits of, respectively, taddpole and sef-energy in the zero momentum limit. \\
qSecond kind will be to calculate approach from \ref{finite momentum}in the zero momentum limit.\\
Later we will discus some "potential only" version with different conditions and discuss whether 
adding finite momentum to it will produce satisfying results.  

\subsubsection{Potential only version}
For comparison, we include also a version of this aprouch steming from \ref{mass_term}.
Inclusion of the mass term do not change the
form of $\dl$ and $\dm$. The frist difference occurs in the potential. \\
First we will describe the case with derivatives II and IV used in renormalisation conditions. 
Then the potential is equal:
\begin{align}
V_R = \cl\lambda\varphi_1^4 + \cm m^2\varphi_1^2 +
\frac{e^4\varphi_1^4}{64\pi^2}\Big(3\log\frac{\varphi_1^2}{v^2}- 
\frac{25}{2}\Big)+
\frac{27e^4v^2\varphi_1^2}{32\pi^2}.
\end{align}
Now, we have that $M_P = 12\cl\lambda v^2 + 2\cm m^2$. 
Now, we will use the condition, that:
\begin{align}\label{tree_first}
\DphiI{}\VT\Big|_v = 0.
\end{align}
We have that $4\cl\lambda v^3+2\cm m^2v = 0$, so $\lambda = -\frac{\cm m^2}{2\cl v^2}$, so 
\begin{align}
m^2&=\frac{-M_P^2}{4\cm} \textrm{\ \ and}\\ 
\lambda &= \frac{M_P^2}{8\cl v^2}. 
\end{align}
Writing $V_R$ with respect to that gives:
\todo{pytanie}
\begin{align}
V_R = \frac{M_P^2\varphi_1^4}{8v^2}-\frac{M_P^2\varphi_1^2}{4} + 
\frac{e^4\varphi_1^4}{64\pi^2}\Big(3\log
\frac{\varphi_1^2}{v^2}- 
\frac{25}{2}\Big)+
\frac{27e^4v^2\varphi_1^2}{32\pi^2}.
\end{align}
From this we have, that:
\begin{equation}
\frac{e^4v^3}{\pi^2} = 0,
\end{equation}
which is also a not safisying result. \\
However, if we drop the condition, that 
$\DphiI{}\VT\Big|_v = 0$, we have potential in the form:
\begin{align}
V_R = \frac{M_P^2-2\cm m^2}{12v^2}\varphi_1^4+\cm m^2\varphi_1^2 + 
\frac{e^4\varphi_1^4}{64\pi^2}\Big(3\log
\frac{\varphi_1^2}{v^2}- 
\frac{25}{2}\Big)+
\frac{27e^4v^2\varphi_1^2}{32\pi^2}.
\end{align}
From thism, using the condition that $\DphiI{}\VR\Big|_v = 0$, we can derive the 
correspondence between $e$ and $M_P$, $v$ and $m$:
\begin{equation}
e^4 = -\frac{(M_P^2+4\cm m^2)\pi^2}{3v^2},
\end{equation}
which is finally a sensible result as it can be realised with real, positive $e$. However, then 
it must hold that $m^2<-\frac{M_P^2}{4\cm}$.
