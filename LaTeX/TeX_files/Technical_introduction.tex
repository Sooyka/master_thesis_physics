% mainfile: ../master_thesis_GW.tex
\chapter{Technical introduction}
\section{Models}
\subsection{Toy model}
%\subsubsection{Conformal toy model}
This model will be used throughout the whole thesis, unless stated otherwise. \\
For a toy model we choose theory of scalar electrodynamics, described by the Lagrangian:
\begin{equation}
\mathcal{L} = -\frac{1}{4}F_{\mu\nu}F^{\mu\nu}+D_\mu\Phi D^\mu\Phi^\dag-\lambda\Phi^4,
\end{equation}
where $\Phi$ is a complex scalar field and the vector field present is $U(1)$ gauge boson. \\

%There will be also version included for comparison, with the mass term explicitly present:
%\begin{equation}
%\mathcal{L} = -\frac{1}{4}F_{\mu\nu}F^{\mu\nu}+D_\mu\Phi D^\mu\Phi^\dag-m^2\Phi^2-\lambda\Phi^4,
%\end{equation}
%The $\cl$, $\cm$ are numerical constants equal to, respectively, $\frac{1}{4}$ and $\frac{1}{2}$ 
%thought the thesis if not stated otherwise. \\

Writing operator $D$ more explicitly it reads:
\begin{equation}
\mathcal{L} = -\frac{1}{4}F_{\mu\nu}F^{\mu\nu}+(\partial_\mu\Phi + ieA_\mu\Phi) 
(\partial^\mu\Phi^\dag-ieA^\mu\Phi^\dag)-\lambda\Phi^4,
\end{equation}

For the reasons that will be clear in ref{finite momentum} we will write $\Phi$ 
field as two real scalar fields 
$\varphi_1$ and $\varphi_2$, such that:
\begin{equation}
\Phi = \frac{1}{\sqrt{2}}(\varphi_1+\varphi_2)
\end{equation}
Then Lagrangian takes form:
\begin{align}
\mathcal{L} = -\frac{1}{4}F_{\mu\nu}F^{\mu\nu}+ \frac{1}{2}(\partial_\mu\varphi_1 - eA_\mu\varphi_2)
(\partial^\mu\varphi_1 - eA^\mu\varphi_2)  \\
+\frac{1}{2}(\partial_\mu\varphi_2+eA_\mu\varphi_1)(\partial^\mu\varphi_2+eA^\mu\varphi_1) 
-\frac{1}{4}\lambda(\varphi_1^2+\varphi_2^2)^2,\notag
\end{align}

which we will write for brevity as:
\begin{align}
\mathcal{L} = -\frac{1}{4}F_{\mu\nu}F^{\mu\nu}+ 
\frac{1}{2}(\partial_\mu\varphi_1 - eA_\mu\varphi_2)^2\\
+\frac{1}{2}(\partial_\mu\varphi_2+eA_\mu\varphi_1)^2
-\frac{1}{4}\lambda(\varphi_1^2+\varphi_2^2)^2. \notag
\end{align}


For a better track of what is independent of numerical convention, we will also write:
\begin{align}\label{scalar_electrodynamics}
\mathcal{L} = -\frac{1}{4}F_{\mu\nu}F^{\mu\nu}+ 
\frac{1}{2}(\partial_\mu\varphi_1 - eA_\mu\varphi_2)^2\\
+\frac{1}{2}(\partial_\mu\varphi_2+eA_\mu\varphi_1)^2
-\cl\lambda(\varphi_1^2+\varphi_2^2)^2, \notag
\end{align}
but $\cl = \frac{1}{4}$ everywhere in the thesis if not stated otherwise.
% \subsubsection{Toy model with explicit mass term}
% \begin{align}\label{mass_term_lagrangian_int}
% \mathcal{L} = &-\frac{1}{4}F_{\mu\nu}F^{\mu\nu}+ 
% \frac{1}{2}(\partial_\mu(\varphi_1+v) - eA_\mu\varphi_2)^2\\
% &+\frac{1}{2}(\partial_\mu\varphi_2+eA_\mu(\varphi_1+v))^2 - \cm m^2((\varphi_1+v)^2+\varphi_2^2)
% -\cl\lambda((\varphi_1+v)^2+\varphi_2^2)^2. \notag
% \end{align}
\subsection{Real model}
$U(2)\times U(2)$ cośtam cośtam

\section{Renormalisation schemes}
\subsection{\texorpdfstring{\MSbar}{MS-bar}}
The minimal-substraction scheme that favorises computation simplicity. \\
The standard structure on one loop is \\
constants are always the same. \\
It is bind to dimentional regularisation.
\subsection{On-shell}
\subsubsection{Zero momentum limit version}
\subsection{Half \texorpdfstring{\MSbar}{MS-bar}-Half On-shell}
\section{Effective potential}

\todo{some statements about effective potential in general} \\
%For now we will be working with our toy model described by Lagrangian \ref{scalar_electrodynamics}
%In this model tree level effective potential is equal to:
%Now, we will calculate

