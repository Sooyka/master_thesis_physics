% mainfile: ../master_thesis_GW.tex
\chapter{On shell renormalisation of the effective potential -- finite momentum approach}

One of the main topic of this thesis is to show a coherent way to renormalise a conformal theory 
without explicit mass term in the on-shell scheme. Along the way, we will also discuss 
the case with explicit mass term, for comparison.\\

%\section{Finite momentum approach}\label{finite momentum}
To calculate on-shell renormalisation we need to calculate self energy. 
However, it turns out, that simple calculation of self energy fails the test of 
comparison between the zero-momentum limit of the self energy and the second derivative 
of the effective potential. \\
Namely, it schould be safisfied that:
\begin{equation}\label{second_derivativ_condition}
\lim\limits_{p^2\to 0}\Sigma(p^2)=\frac{\partial^2V_{eff}}{\partial\varphi_1^2},
\end{equation}
\todo{napisać ile wychodzi}
but it is not the case. \\

However, from the \MSbar considerations, we know that $\Phi$ have non-zero VEV, let us 
call it $v$. 
Let us rotate $\Phi$ in such a way, that $\langle\varphi_1\rangle=v$ and $\langle\varphi_2\rangle 
= 0$, where now $v$ is real. \\
Keeping this in mind, we can rewrite Lagrangian in terms of shifted fields $\varphi_1$, 
$\varphi_2$ which have both zero VEV, now VEV is explicitly in the Lagrangian:
\begin{align}\label{no_mass_term}
\mathcal{L} = -\frac{1}{4}F_{\mu\nu}F^{\mu\nu}+ 
\frac{1}{2}(\partial_\mu(\varphi_1+v) - eA_\mu\varphi_2)^2\\
+\frac{1}{2}(\partial_\mu\varphi_2+eA_\mu(\varphi_1+v))^2
-\cl\lambda((\varphi_1+v)^2+\varphi_2^2)^2. \notag
\end{align}

This breaks the symmetry, but now there are more interaction terms in the Lagrangian and this leads 
to different self energy, now consistent with the second derivative of the effective potential, 
as will be shown in \ref{}. \\ 

Following \cite{Coleman1973} we put the mass counterterm even though initialy the mass term 
was not present in the Lagrangian. It will turn out to be crutial in \ref{}. \\ 
The Lagrangian with $\dZ$, $\dl$ and $\dm$ counterterms looks like this:

\begin{align}
\mathcal{L_R} = &-\frac{1}{4}F_{\mu\nu}F^{\mu\nu} \\
&+(1+ \dZ)(\frac{1}{2}(\partial_\mu(\varphi_1+v) - eA_\mu\varphi_2)^2
+\frac{1}{2}(\partial_\mu\varphi_2+eA_\mu(\varphi_1+v))^2) \notag \\
&-(1+\dZ)^2\cl(\lambda+\dl)((\varphi_1+v)^2+\varphi_2^2)^2  \notag \\
&-c_m\dm((\varphi_1+v)^2+\varphi_2^2).
\end{align}

%We will write with the convention that:
%\begin{equation}
%\dl=\dZl\lambda
%\end{equation}

Separating the terms with the first power of renormalisation constants and second power of 
$\varphi_1$, we obtain correction to the self energy equal to:
\begin{equation}
\dS=-12\cl v^2(2\lambda\dZphi+\dl)-2\cm\dm-p^2\dZphi,
\end{equation}

where $p^2 = -\partial_\mu \varphi_1\partial^\mu\varphi_1$.


Separating the terms with the first power of renormalisation constants and first power of 
$\varphi_1$, we obtain correction to the tadpole equal to:
\begin{equation}
\dT=-4\cl v^3(2\lambda\dZphi+\dl)-2\cm\dm v
\end{equation}



First approach is to impose renormalisation conditions resembling classical on-shell.
Here, $\Sigma'$ stands for $\frac{\textrm{d}\Sigma}{\textrm{d}p^2}$ and, if not 
stated otherwise, 
$\Sigma$, $\dS$ and $\Sigma'$ are evaluated at $p^2 = M_P^2$, where 
$M_P$ stands for physical mass. We denote real part as $\Rep{}$.
\begin{align}
T+\dT &= 0 \\
\Rep{\Sigma} + \Rep{\dS} &= 0 \\
\Rep{\Sigma'} &= 0
\end{align}

This gives us:
\begin{align}
\dm &= \frac{-1}{4\cm}\left(\Rep{\Sigma}-\frac{3}{v}T-M_P^2\Rep{\Sigma'}
\right)\\
\dl &= \frac{1}{8\cl v^2}\left(\Rep{\Sigma}-\frac{1}{v}T-(16\cl\lambda v^2+M_P^2)
\Rep{\Sigma'}
\right)\\
\dZ &= \Rep{\Sigma'}
\end{align}

We define $M_P^2$ as the second derivative of the tree potential evaluated at VEV, so:
\begin{equation}
M_P^2 = \frac{\partial^2}{\partial \varphi_1^2}\cl\lambda\varphi_1^4\Big|_{v} = 12\cl\lambda v^2
\end{equation}
and we impose, that it does not change after one loop contributions. \\

Contributions to $\Sigma$ and $T$ constitutes of the following diagrams:
\todo{diagrams}

The values of diagrams are as follows:
Contributing to $\Sigma$:
\begin{align}
&-i\frac{e^2}{M_V}\Big[M_V^2a(M_2)+(-p^2-M_V^2+M_2^2)a(M_V)-(p^2+M_2^2)^2b_0(p,0,M_2)+ 
\notag\\ 
&(p^2+M_2^2-M_V^2)^2b_0(p,M_V,M_2)\Big]
\end{align}
\begin{align}
&-i\frac{e^4v^2}{2M_V^4}\Big[2M_V^2a(M_V)+p^4b_0(p,0,0)-2(p^2-M_V^2)^2b_0(p,M_V,0)+\notag \\
&16M_V^4b_0^b(p,M_V,M_V)+(p^4-4p^2M_V^2-4M_V^4)b_0(p,M_V,M_V)\Big]
\end{align}
\begin{align}
-i3e^2a_b(M_V) \\
-i12\cl\lambda a(M_1) \\
-i4\cl\lambda a(M_2) \\
-i288\cl^2\lambda^2v^2b_0(p,M_1,M_1) \\
-i32\cl^2\lambda^2 v^2b_0(p,M_2,M_2) \\
\end{align}
Contributing to $T$:
\begin{align}
-i3e^2va^b(M_V) \\
-i12\cl\lambda va(M_1) \\
-i4\cl\lambda va(M_2) \\
-i4\cl\lambda v^3
\end{align}
Where
\begin{align}
a(M) = \\
b_0(p,M_1,M_2) = \\
a^b(M) = \\
b_0^b(p,M_1,M_2) = 
\end{align}

Here, we will be interested in only contributions up to order $e^4$. From our \MSbar considerations 
we can see, that $\lambda$ should be of order $e^4$, that $M_V^2$ should be of order $e^2$ and 
that $M_1$, $M_2$ should be of order $e^4$. 
As that we are 
intersted only in following parts of contributions to $\Sigma$ and $T$:
\begin{align}
\Sigma_{e^0} = &-\frac{e^2}{M_V^2}\Big[-p^4b_0(p,0,M_2)+p^4b_0(p,M_V,M_2)\Big]-\notag \\
&\frac{e^4v^2}{M_V^4}\Big[p^4b_0(p,0,0)-2p^4b_0(p,M_V,0)+p^4b_0(p,M_V,M_V)\Big] \\
\Sigma_{e^2} = &-\frac{e^2}{M_V^2}\Big[-p^2a(M_V)-2p^2M_V^2b_0(p,M_V,M_2)\Big]-\notag \\
&\frac{e^4v^2}{2M_V^4}\Big[4p^2M_V^2b_0(p,M_V,0)-4p^2M_V^2b_0(p,M_V,M_V)\Big] \\
\Sigma_{e^4} = &-\frac{e^2}{M_V^2}a(M_V)\Big[-M_V^2a(M_V)+M_V^4b_0(p,M_V,M_2)\Big]-\notag \\
&\frac{e^4v^2}{2M_V^4}\Big[2M_V^2a(M_V)-2M_V^4b_0(p,M_V,0)+\notag \\
&16M_V^4b_0^b(p,M_V,M_V)-4M_V^4b_0(p,M_V,M_V)\Big]-\notag\\
&3e^2a^b(M_V)-\notag\\
&\frac{e^2}{M_V^2}\Big[-2p^2M_2^2b_0(p,0,M_2)+2p^2M_2^2b_0(p,M_V,M_2)\Big] \\
T_{e^4} = & -3e^2va^b(M_V)-4\cl\lambda v^3
\end{align}
The divergent part of $T$, $\Sigma$ and $\Sigma'$ are:
\begin{align}
\textrm{div}T = -\frac{3e^4v^4}{16\pi^2}\Big(-\frac{2}{\epsilon}\Big)\\
\textrm{div}\Sigma = \frac{6e^2(M_P^2-3e^2v^2)}{32\pi^2}\Big(-\frac{2}{\epsilon}\Big)\\
\textrm{div}\Sigma' = \frac{3e^2}{16\pi^2}\Big(-\frac{2}{\epsilon}\Big)
\end{align}
After subsituing to $\dl$, $\dm$, $\dZ$ and then to $V_R$ we see that $\textrm{div}V_R = 0$, 
thus renormalisation procedure succeeds in canceling divergences.
%\subsection{Explicit mass term case}
%For the comparison, we will present the same calculation, performed on the analogous theory with 
%explicit mass term. Similarly as in the \ref{no_mass_term} we need to shift fields for 
%\ref{second_derivativ_condition} to be satisfied. The Lagrangian in this case is:
%\begin{align}\label{mass_term}
%\mathcal{L} = &-\frac{1}{4}F_{\mu\nu}F^{\mu\nu}+ 
%\frac{1}{2}(\partial_\mu(\varphi_1+v) - eA_\mu\varphi_2)^2\\
%&+\frac{1}{2}(\partial_\mu\varphi_2+eA_\mu(\varphi_1+v))^2 - \cm m^2((\varphi_1+v)^2+\varphi_2^2)
%-\cl\lambda((\varphi_1+v)^2+\varphi_2^2)^2. \notag
%\end{align}
%With renormalisation constants:
%\begin{align}
%\mathcal{L_R} = &-\frac{1}{4}F_{\mu\nu}F^{\mu\nu} \\
%&+(1+ \dZ)(\frac{1}{2}(\partial_\mu(\varphi_1+v) - eA_\mu\varphi_2)^2
%+\frac{1}{2}(\partial_\mu\varphi_2+eA_\mu(\varphi_1+v))^2) \notag \\
%&-(1+\dZ)^2\cl(\lambda+\dl)((\varphi_1+v)^2+\varphi_2^2)^2  \notag \\
%&-(1+\dZ)c_m(m+\dm)((\varphi_1+v)^2+\varphi_2^2).
%\end{align}
%Corrections then are:
%\begin{align}
%\dS&=-12\cl v^2(2\lambda\dZphi+\dl)-2\cm\dm-2\cm m^2\dZ -p^2\dZ \\
%\dT&=-4\cl v^3(2\lambda\dZphi+\dl)-2\cm\dm v-2\cm m^2v\dZ.
%\end{align}
%This changes the form of renormalisation constants to:
%\begin{align}
%\dm &= \frac{-1}{4\cm}\left(\Rep{\Sigma}-\frac{3}{v}T-(4\cm m^2+M_P^2)\Rep{\Sigma'}
%\right)\label{delta_m_mass}\\
%\dl &= \frac{1}{8\cl v^2}\left(\Rep{\Sigma}-\frac{1}{v}T-(16\cl\lambda v^2+M_P^2)
%\Rep{\Sigma'}
%\right)\\
%\dZ &= \Rep{\Sigma'}
%\end{align}
%The only difference is $4\cm m^2$ term in \ref{delta_m_mass}. 
%\section{"Zero momentum" approach}
%Here we will compare two kinds of "zero momentum" approach. 
%First will be imposing renormalisation conditions in terms of only derivatives of 
%effective potential. This is the zero momentum approach as first and second derivatives 
%are limits of, respectively, taddpole and sef-energy in the zero momentum limit. \\
%qSecond kind will be to calculate approach from \ref{finite momentum}in the zero momentum limit.\\
%Later we will discus some "potential only" version with different conditions and discuss whether 
%adding finite momentum to it will produce satisfying results.  














