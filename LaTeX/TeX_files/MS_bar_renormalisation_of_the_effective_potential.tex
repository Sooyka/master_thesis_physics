% mainfile: ../master_thesis_GW.tex
\chapter{MS bar renormalisation of the effective potential}



\MSbar renormalisation gives:
\begin{align}
V_{R}=\cl\lambda\varphi_1^4 + \frac{1}{4}\frac{(\frac{1}{2}\lambda\varphi_1^2)^2}{(4\pi)^2}&\left(-\frac{3}{2}+ 
\log{\frac{(\frac{1}{2}\lambda\varphi_1^2)^2}{\mu^2}}\right) + \\
\frac{1}{4}\frac{(\frac{1}{6}\lambda\varphi_1^2)^2}{(4\pi)^2}&\left(-\frac{3}{2}+ 
\log{\frac{(\frac{1}{6}\lambda\varphi_1^2)^2}{\mu^2}}\right) + \\
\frac{1}{4}\frac{3(e^2\varphi_1^2)^2}{(4\pi)^2}&\left(-\frac{5}{6}+
\log{\frac{e^2\varphi_1^2}{\mu^2}}\right)
\end{align}

Here, as will be apparent in the second, we are interested only in $e^4$ part, so from now on, 
we will write:
\begin{align}
V_{R}=\cl\lambda\varphi_1^4 + \frac{1}{4}\frac{3(e^2\varphi_1^2)^2}{(4\pi)^2}&\left(-\frac{5}{6}+
\log{\frac{e^2\varphi_1^2}{\mu^2}}\right)
\end{align}

We can bind $e$ to $\lambda$ and $v$ at the loop level, demanding that VEV does not change 
due to one loop corrections, stating that:
\begin{align}
\DphiI\VR \Big|_v= 0
\end{align}
This gives the condition:
\begin{align}
4\cl\lambda v^3-\frac{e^4v^3}{16\pi^2}-\frac{3e^4v^3}{16\pi^2}\log\frac{e^2v^2}{\mu^2}=0
\end{align}
Setting scale parameter $\mu$ to the effective mass of the vector, namely $ev$, we have 
simpler form of:
\begin{align}
4\cl\lambda v^3-\frac{e^4v^3}{16\pi^2}=0
\end{align}
Which gives:
\begin{align}\label{e4_lambda_MS}
\lambda = \frac{e^4}{64\cl\pi^2}
\end{align}
Writing potential with this substitutions yields:

\begin{align}
V_R = \frac{3e^4}{64\pi^2}\Big(-\frac{1}{2}+\log\frac{\varphi_1^2}{v^2}\Big)\varphi_1^4
\end{align}


Equations \ref{e4_lambda_MS} is very important in this discussion as it states, that $\lambda$ 
is of order $e^4$ in our model. This post factum justifies our 
choice in taking only $e^4$ part, as other part was of order $e^8$. \\

We can now also bind square if the physical mass $M_P^2$ as the second derivative of the 
renormalised effective potential at VEV.

\begin{align}
M_P^2 = \DphiII\VR \Big|_v = \frac{3e^4}{16\pi^2}\Big(-\frac{1}{2}\Big)v^2+\frac{e^4v^2}{8\pi^2} 
+\frac{3e^4v^2}{32\pi^2}
\end{align}
This gives that:
\begin{align}
M_P^2 = \frac{e^4v^2}{8\pi^2}
\end{align}

From this we have that the ratio between scalar mass $M_P^2$ and vector mass $m(V)^2$ is:
\begin{align}
\frac{M_P^2}{m(V)^2} = \frac{\frac{e^4v^2}{8\pi^2}}{e^2v^2} = \frac{e^2}{8\pi^2}
\end{align}

We can now also express $e^4$ and $\lambda$ in terms of $M_P^2$ and $v$:
\begin{align}
e^4 = \frac{8M_P^2\pi^2}{v^2} \\
\lambda = \frac{M_P^2}{8\cl v^2}
\end{align}

Writing potential in these terms gives:
\begin{align}
V_R = \frac{3M_P^2}{8v^2}\Big(-\frac{1}{2}+\log\frac{\varphi_1^2}{v^2}\Big)\varphi_1^4
\end{align}


