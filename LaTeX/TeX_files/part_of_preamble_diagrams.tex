% mainfile: ../master_thesis_GW.tex
\usepackage{graphicx}
\usepackage{slashed}            % for slashed characters in math mode
\usepackage{bbm}                % for \mathbbm{1} (unit matrix)
\usepackage{xspace}				% For spacing after commands

\setcounter{MaxMatrixCols}{30}
\numberwithin{equation}{section} \setlength{\textwidth}{17.5cm}
\setlength{\textheight}{22.5cm} \setlength{\oddsidemargin}{-0.5cm}
\setlength{\evensidemargin}{1cm} \setlength{\headheight}{0cm}
\setlength{\headsep}{0cm} \setlength{\topmargin}{0cm}
\setlength{\footskip}{1.5cm} \baselineskip 0.6cm


% TIKZ - for drawing Feynman diagrams
% ... use with pdflatex

%\usepackage{tikz}
%\usetikzlibrary{arrows,shapes}
%\usetikzlibrary{trees}
%\usetikzlibrary{matrix,arrows} 				% For commutative diagram
%											% http://www.felixl.de/commu.pdf
%\usetikzlibrary{positioning}				% For "above of=" commands
%\usetikzlibrary{calc,through}				% For coordinates
%\usetikzlibrary{decorations.pathreplacing}  % For curly braces
%% http://www.math.ucla.edu/~getreuer/tikz.html
%\usepackage{pgffor}							% For repeating patterns

%\usetikzlibrary{decorations.pathmorphing}	% For Feynman Diagrams
%\usetikzlibrary{decorations.markings}
%\tikzset{
%	% >=stealth', %%  Uncomment for more conventional arrows
%    vector/.style={decorate, decoration={snake}, draw},
%	provector/.style={decorate, decoration={snake,amplitude=2.5pt}, draw},
%	antivector/.style={decorate, decoration={snake,amplitude=-2.5pt}, draw},
%    fermion/.style={draw=black, postaction={decorate},
%        decoration={markings,mark=at position .55 with {\arrow[draw=black]{>}}}},
%    fermionbar/.style={draw=black, postaction={decorate},
%        decoration={markings,mark=at position .55 with {\arrow[draw=black]{<}}}},
%    fermionnoarrow/.style={draw=black},
%    gluon/.style={decorate, draw=black,
%        decoration={coil,amplitude=4pt, segment length=5pt}},
%    scalar/.style={dashed,draw=black, postaction={decorate},
%        decoration={markings,mark=at position .55 with {\arrow[draw=black]{>}}}},
%    scalarbar/.style={dashed,draw=black, postaction={decorate},
%        decoration={markings,mark=at position .55 with {\arrow[draw=black]{<}}}},
%    scalarnoarrow/.style={dashed,draw=black},
%    electron/.style={draw=black, postaction={decorate},
%        decoration={markings,mark=at position .55 with {\arrow[draw=black]{>}}}},
%	bigvector/.style={decorate, decoration={snake,amplitude=4pt}, draw},
%}

%% TIKZ - for block diagrams, 
%% from http://www.texample.net/tikz/examples/control-system-principles/
%% \usetikzlibrary{shapes,arrows}
%\tikzstyle{block} = [draw, rectangle, 
%    minimum height=3em, minimum width=6em]
    
    
\usepackage{tikz}
\usetikzlibrary{arrows,shapes}
\usetikzlibrary{trees}
\usetikzlibrary{matrix,arrows} 				% For commutative diagram
											% http://www.felixl.de/commu.pdf
\usetikzlibrary{positioning}				% For "above of=" commands
\usetikzlibrary{calc,through}				% For coordinates
\usetikzlibrary{decorations.pathreplacing}  % For curly braces
% http://www.math.ucla.edu/~getreuer/tikz.html
\usepackage{pgffor}							% For repeating patterns

\usetikzlibrary{decorations.pathmorphing}	% For Feynman Diagrams
\usetikzlibrary{decorations.markings}
\tikzset{
	% >=stealth', %%  Uncomment for more conventional arrows
    vector/.style={decorate, decoration={snake}, draw},
	provector/.style={decorate, decoration={snake,amplitude=2.5pt}, draw},
	antivector/.style={decorate, decoration={snake,amplitude=-2.5pt}, draw},
    fermion/.style={draw=black, postaction={decorate},
        decoration={markings,mark=at position .55 with {\arrow[draw=black]{>}}}},
    fermionbar/.style={draw=black, postaction={decorate},
        decoration={markings,mark=at position .55 with {\arrow[draw=black]{<}}}},
    fermionnoarrow/.style={draw=black},
    gluon/.style={decorate, draw=black,
        decoration={coil,amplitude=4pt, segment length=5pt}},
    scalararrow/.style={dashed,draw=black, postaction={decorate},
        decoration={markings,mark=at position .55 with {\arrow[draw=black]{>}}}},
    scalar/.style={dashed,draw=black},
    scalarbar/.style={dashed,draw=black, postaction={decorate},
        decoration={markings,mark=at position .55 with {\arrow[draw=black]{<}}}},
    %scalarnoarrow/.style={dashed,draw=black},
    electron/.style={draw=black, postaction={decorate},
        decoration={markings,mark=at position .55 with {\arrow[draw=black]{>}}}},
	bigvector/.style={decorate, decoration={snake,amplitude=4pt}, draw},
}

% TIKZ - for block diagrams, 
% from http://www.texample.net/tikz/examples/control-system-principles/
% \usetikzlibrary{shapes,arrows}
\tikzstyle{block} = [draw, rectangle, 
    minimum height=3em, minimum width=6em]


